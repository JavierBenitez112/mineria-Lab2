% Options for packages loaded elsewhere
\PassOptionsToPackage{unicode}{hyperref}
\PassOptionsToPackage{hyphens}{url}
\documentclass[
]{article}
\usepackage{xcolor}
\usepackage[margin=1in]{geometry}
\usepackage{amsmath,amssymb}
\setcounter{secnumdepth}{-\maxdimen} % remove section numbering
\usepackage{iftex}
\ifPDFTeX
  \usepackage[T1]{fontenc}
  \usepackage[utf8]{inputenc}
  \usepackage{textcomp} % provide euro and other symbols
\else % if luatex or xetex
  \usepackage{unicode-math} % this also loads fontspec
  \defaultfontfeatures{Scale=MatchLowercase}
  \defaultfontfeatures[\rmfamily]{Ligatures=TeX,Scale=1}
\fi
\usepackage{lmodern}
\ifPDFTeX\else
  % xetex/luatex font selection
\fi
% Use upquote if available, for straight quotes in verbatim environments
\IfFileExists{upquote.sty}{\usepackage{upquote}}{}
\IfFileExists{microtype.sty}{% use microtype if available
  \usepackage[]{microtype}
  \UseMicrotypeSet[protrusion]{basicmath} % disable protrusion for tt fonts
}{}
\makeatletter
\@ifundefined{KOMAClassName}{% if non-KOMA class
  \IfFileExists{parskip.sty}{%
    \usepackage{parskip}
  }{% else
    \setlength{\parindent}{0pt}
    \setlength{\parskip}{6pt plus 2pt minus 1pt}}
}{% if KOMA class
  \KOMAoptions{parskip=half}}
\makeatother
\usepackage{color}
\usepackage{fancyvrb}
\newcommand{\VerbBar}{|}
\newcommand{\VERB}{\Verb[commandchars=\\\{\}]}
\DefineVerbatimEnvironment{Highlighting}{Verbatim}{commandchars=\\\{\}}
% Add ',fontsize=\small' for more characters per line
\usepackage{framed}
\definecolor{shadecolor}{RGB}{248,248,248}
\newenvironment{Shaded}{\begin{snugshade}}{\end{snugshade}}
\newcommand{\AlertTok}[1]{\textcolor[rgb]{0.94,0.16,0.16}{#1}}
\newcommand{\AnnotationTok}[1]{\textcolor[rgb]{0.56,0.35,0.01}{\textbf{\textit{#1}}}}
\newcommand{\AttributeTok}[1]{\textcolor[rgb]{0.13,0.29,0.53}{#1}}
\newcommand{\BaseNTok}[1]{\textcolor[rgb]{0.00,0.00,0.81}{#1}}
\newcommand{\BuiltInTok}[1]{#1}
\newcommand{\CharTok}[1]{\textcolor[rgb]{0.31,0.60,0.02}{#1}}
\newcommand{\CommentTok}[1]{\textcolor[rgb]{0.56,0.35,0.01}{\textit{#1}}}
\newcommand{\CommentVarTok}[1]{\textcolor[rgb]{0.56,0.35,0.01}{\textbf{\textit{#1}}}}
\newcommand{\ConstantTok}[1]{\textcolor[rgb]{0.56,0.35,0.01}{#1}}
\newcommand{\ControlFlowTok}[1]{\textcolor[rgb]{0.13,0.29,0.53}{\textbf{#1}}}
\newcommand{\DataTypeTok}[1]{\textcolor[rgb]{0.13,0.29,0.53}{#1}}
\newcommand{\DecValTok}[1]{\textcolor[rgb]{0.00,0.00,0.81}{#1}}
\newcommand{\DocumentationTok}[1]{\textcolor[rgb]{0.56,0.35,0.01}{\textbf{\textit{#1}}}}
\newcommand{\ErrorTok}[1]{\textcolor[rgb]{0.64,0.00,0.00}{\textbf{#1}}}
\newcommand{\ExtensionTok}[1]{#1}
\newcommand{\FloatTok}[1]{\textcolor[rgb]{0.00,0.00,0.81}{#1}}
\newcommand{\FunctionTok}[1]{\textcolor[rgb]{0.13,0.29,0.53}{\textbf{#1}}}
\newcommand{\ImportTok}[1]{#1}
\newcommand{\InformationTok}[1]{\textcolor[rgb]{0.56,0.35,0.01}{\textbf{\textit{#1}}}}
\newcommand{\KeywordTok}[1]{\textcolor[rgb]{0.13,0.29,0.53}{\textbf{#1}}}
\newcommand{\NormalTok}[1]{#1}
\newcommand{\OperatorTok}[1]{\textcolor[rgb]{0.81,0.36,0.00}{\textbf{#1}}}
\newcommand{\OtherTok}[1]{\textcolor[rgb]{0.56,0.35,0.01}{#1}}
\newcommand{\PreprocessorTok}[1]{\textcolor[rgb]{0.56,0.35,0.01}{\textit{#1}}}
\newcommand{\RegionMarkerTok}[1]{#1}
\newcommand{\SpecialCharTok}[1]{\textcolor[rgb]{0.81,0.36,0.00}{\textbf{#1}}}
\newcommand{\SpecialStringTok}[1]{\textcolor[rgb]{0.31,0.60,0.02}{#1}}
\newcommand{\StringTok}[1]{\textcolor[rgb]{0.31,0.60,0.02}{#1}}
\newcommand{\VariableTok}[1]{\textcolor[rgb]{0.00,0.00,0.00}{#1}}
\newcommand{\VerbatimStringTok}[1]{\textcolor[rgb]{0.31,0.60,0.02}{#1}}
\newcommand{\WarningTok}[1]{\textcolor[rgb]{0.56,0.35,0.01}{\textbf{\textit{#1}}}}
\usepackage{graphicx}
\makeatletter
\newsavebox\pandoc@box
\newcommand*\pandocbounded[1]{% scales image to fit in text height/width
  \sbox\pandoc@box{#1}%
  \Gscale@div\@tempa{\textheight}{\dimexpr\ht\pandoc@box+\dp\pandoc@box\relax}%
  \Gscale@div\@tempb{\linewidth}{\wd\pandoc@box}%
  \ifdim\@tempb\p@<\@tempa\p@\let\@tempa\@tempb\fi% select the smaller of both
  \ifdim\@tempa\p@<\p@\scalebox{\@tempa}{\usebox\pandoc@box}%
  \else\usebox{\pandoc@box}%
  \fi%
}
% Set default figure placement to htbp
\def\fps@figure{htbp}
\makeatother
\setlength{\emergencystretch}{3em} % prevent overfull lines
\providecommand{\tightlist}{%
  \setlength{\itemsep}{0pt}\setlength{\parskip}{0pt}}
\usepackage{bookmark}
\IfFileExists{xurl.sty}{\usepackage{xurl}}{} % add URL line breaks if available
\urlstyle{same}
\hypersetup{
  pdftitle={Lab Aprendizaje No Supervisado},
  pdfauthor={Pedro Avila, Javier Benitez, Brandon Rivera},
  hidelinks,
  pdfcreator={LaTeX via pandoc}}

\title{Lab Aprendizaje No Supervisado}
\author{Pedro Avila, Javier Benitez, Brandon Rivera}
\date{2026-02-07}

\begin{document}
\maketitle

\begin{Shaded}
\begin{Highlighting}[]
\CommentTok{\# dependecies}
\CommentTok{\#install.packages("dplyr")}
\end{Highlighting}
\end{Shaded}

\section{Librerias}\label{librerias}

\begin{Shaded}
\begin{Highlighting}[]
\FunctionTok{library}\NormalTok{(ggplot2)}
\FunctionTok{library}\NormalTok{(dplyr)}
\FunctionTok{library}\NormalTok{(nortest)}
\FunctionTok{library}\NormalTok{(scales)}
\FunctionTok{library}\NormalTok{(lubridate)}
\FunctionTok{library}\NormalTok{(cluster)}
\FunctionTok{library}\NormalTok{(factoextra)}
\FunctionTok{library}\NormalTok{(arules)}
\FunctionTok{library}\NormalTok{(arulesViz)}
\FunctionTok{library}\NormalTok{(hopkins)}
\FunctionTok{library}\NormalTok{(fpc)}
\FunctionTok{library}\NormalTok{(psych)}
\FunctionTok{library}\NormalTok{(corrplot)}
\FunctionTok{library}\NormalTok{(tidyr)}
\end{Highlighting}
\end{Shaded}

\section{Exploracion de datos}\label{exploracion-de-datos}

\begin{itemize}
\tightlist
\item
  Una exploración rápida de sus datos, para eso haga un resumen de su
  conjunto de datos.
\end{itemize}

\begin{Shaded}
\begin{Highlighting}[]
\NormalTok{movies }\OtherTok{\textless{}{-}} \FunctionTok{read.csv}\NormalTok{(}\StringTok{"Movies\_2026.csv"}\NormalTok{, }\AttributeTok{fileEncoding =} \StringTok{"latin1"}\NormalTok{)}

\FunctionTok{str}\NormalTok{(movies)}
\end{Highlighting}
\end{Shaded}

\begin{verbatim}
## 'data.frame':    19883 obs. of  28 variables:
##  $ id                       : int  1627085 1626914 1626898 1626808 1626678 1626234 1626010 1625551 1625043 1624457 ...
##  $ budget                   : num  0 0 0 0 0 1 0 0 0 0 ...
##  $ genres                   : chr  "Drama|Crime" "Animation" "Animation" "Thriller|Mystery|Documentary" ...
##  $ homePage                 : chr  "" "" "" "" ...
##  $ productionCompany        : chr  "" "" "" "" ...
##  $ productionCompanyCountry : chr  "" "" "" "" ...
##  $ productionCountry        : chr  "" "" "" "" ...
##  $ revenue                  : num  0 0 0 0 0 1 0 0 0 0 ...
##  $ runtime                  : int  95 3 2 5 12 14 39 90 96 106 ...
##  $ video                    : logi  FALSE FALSE FALSE FALSE FALSE FALSE ...
##  $ director                 : chr  "Javad Hakami" "Kimmy Gatewood" "Kimmy Gatewood" "Felipe Roldán" ...
##  $ actors                   : chr  "Mohsen Ghasabian|Aida Mahiani|Mehran Ghafourian|Payam Ahmadinia|Masoud Karamati|Roya Javidnia|Nasim Adabi|Siavash Cheraghipour" "Kameron Jackson|Laura Weaving|sara weaving|Bertha Williams" "Cedric Mitchell|Cajun mills|Laura Williams" "Tomás Tuchsznajder|Matias Junas|Martin Etcheverry|Romeo Jeirfimczuk|Agustin Pulido|Alec Drach|Franco Serio" ...
##  $ actorsPopularity         : chr  "0.3453|0.1664|0.9684|0.3437|0.3713|0.2437|0.2796|0.2639" "0|0.0071|0|0" "0.0193|0|0.0143" "0|0|0|0|0|0|0" ...
##  $ actorsCharacter          : chr  "|||||||" "Prince Charming|Evil Stepmother|Fairy Godmother|Cinderella" "Aladdin|Jafar|Jasmine" "||||||" ...
##  $ originalTitle            : chr  "غÙ\u0088Ø·Ù\u0087 Ù\u0088ر" "Cinderella" "Aladdin" "EL ANILLO Y EL DECK" ...
##  $ title                    : chr  "Immersed" "Cinderella" "Aladdin" "THE RING AND THE DECK" ...
##  $ originalLanguage         : chr  "fa" "en" "en" "es" ...
##  $ popularity               : num  0.0357 0.0357 0.0214 0.0429 0.0379 ...
##  $ releaseDate              : chr  "2026-02-01" "2026-02-01" "2026-02-01" "2026-02-01" ...
##  $ voteAvg                  : num  0 0 0 0 0 0 0 0 0 0 ...
##  $ voteCount                : int  0 0 0 0 0 0 0 0 0 0 ...
##  $ genresAmount             : int  2 1 1 3 1 1 1 1 3 1 ...
##  $ productionCoAmount       : int  0 0 0 0 0 0 0 0 0 0 ...
##  $ productionCountriesAmount: int  0 0 0 0 0 0 0 1 1 0 ...
##  $ actorsAmount             : int  8 4 3 7 3 3 5 4 5 5 ...
##  $ castWomenAmount          : int  2 0 0 0 0 0 0 3 1 2 ...
##  $ castMenAmount            : int  5 0 0 0 0 0 3 0 3 3 ...
##  $ releaseYear              : int  2026 2026 2026 2026 2026 2026 2026 2026 2026 2026 ...
\end{verbatim}

\begin{itemize}
\tightlist
\item
  Al momento de usar \texttt{str()}, nos podemos hacer una idea de como
  son los datos en todo el dataset, este analisis se hace luego en el
  proyecto (en especifico en el inciso 2). Como nos podemos dar cuenta
  varios de estos campos llegan a tener informacion vacia para su tipo
  de dato respectivo.
\end{itemize}

\begin{Shaded}
\begin{Highlighting}[]
\NormalTok{movies }\OtherTok{\textless{}{-}} \FunctionTok{subset}\NormalTok{(movies, }\AttributeTok{select =} \SpecialCharTok{{-}}\NormalTok{id) }\CommentTok{\# we ain\textquotesingle{}t going to summarize the the id for pretty obvious reasons}
\FunctionTok{summary}\NormalTok{(movies)}
\end{Highlighting}
\end{Shaded}

\begin{verbatim}
##      budget             genres            homePage         productionCompany 
##  Min.   :        0   Length:19883       Length:19883       Length:19883      
##  1st Qu.:        0   Class :character   Class :character   Class :character  
##  Median :        0   Mode  :character   Mode  :character   Mode  :character  
##  Mean   :  9413280                                                           
##  3rd Qu.:  1000000                                                           
##  Max.   :380000000                                                           
##                                                                              
##  productionCompanyCountry productionCountry     revenue         
##  Length:19883             Length:19883       Min.   :0.000e+00  
##  Class :character         Class :character   1st Qu.:0.000e+00  
##  Mode  :character         Mode  :character   Median :0.000e+00  
##                                              Mean   :2.879e+07  
##                                              3rd Qu.:3.306e+05  
##                                              Max.   :2.847e+09  
##                                                                 
##     runtime         video           director            actors         
##  Min.   :  0.00   Mode :logical   Length:19883       Length:19883      
##  1st Qu.: 10.00   FALSE:19313     Class :character   Class :character  
##  Median : 86.00   TRUE :84        Mode  :character   Mode  :character  
##  Mean   : 66.09   NA's :486                                            
##  3rd Qu.:103.00                                                        
##  Max.   :750.00                                                        
##                                                                        
##  actorsPopularity   actorsCharacter    originalTitle         title          
##  Length:19883       Length:19883       Length:19883       Length:19883      
##  Class :character   Class :character   Class :character   Class :character  
##  Mode  :character   Mode  :character   Mode  :character   Mode  :character  
##                                                                             
##                                                                             
##                                                                             
##                                                                             
##  originalLanguage     popularity        releaseDate           voteAvg      
##  Length:19883       Min.   :0.000e+00   Length:19883       Min.   : 0.000  
##  Class :character   1st Qu.:5.460e-02   Class :character   1st Qu.: 0.000  
##  Mode  :character   Median :8.502e+00   Mode  :character   Median : 5.400  
##                     Mean   :2.625e+01                      Mean   : 3.837  
##                     3rd Qu.:2.224e+01                      3rd Qu.: 6.800  
##                     Max.   :1.147e+04                      Max.   :10.000  
##                                                                            
##    voteCount        genresAmount    productionCoAmount
##  Min.   :    0.0   Min.   : 0.000   Min.   : 0.000    
##  1st Qu.:    0.0   1st Qu.: 1.000   1st Qu.: 0.000    
##  Median :    6.0   Median : 2.000   Median : 1.000    
##  Mean   :  675.9   Mean   : 1.949   Mean   : 1.973    
##  3rd Qu.:  423.0   3rd Qu.: 3.000   3rd Qu.: 3.000    
##  Max.   :30788.0   Max.   :16.000   Max.   :89.000    
##                                                       
##  productionCountriesAmount  actorsAmount    castWomenAmount  castMenAmount   
##  Min.   :  0.00            Min.   :     0   Min.   :     0   Min.   :     0  
##  1st Qu.:  1.00            1st Qu.:     3   1st Qu.:     0   1st Qu.:     0  
##  Median :  1.00            Median :     9   Median :     2   Median :     3  
##  Mean   :  1.23            Mean   :  1082   Mean   :  3517   Mean   :  8224  
##  3rd Qu.:  1.00            3rd Qu.:    21   3rd Qu.:     6   3rd Qu.:    12  
##  Max.   :155.00            Max.   :919590   Max.   :922162   Max.   :922017  
##                                             NA's   :37       NA's   :162     
##   releaseYear  
##  Min.   :1902  
##  1st Qu.:2013  
##  Median :2021  
##  Mean   :2017  
##  3rd Qu.:2025  
##  Max.   :2026  
##  NA's   :2
\end{verbatim}

\begin{itemize}
\tightlist
\item
  Al resumirlos no podemos agarrar informacion tan relevante para hacer
  cualquier tipo de analisis ademas de algunos valores numericos en
  dichos campos. Como ver cantidad de ingresos maximos y medios,
  prespuesto y ganancia por peliculas. No llega a ser informacion tan
  valiosa solo viendola de por si, ya que son datos generales entre
  todas y no de una solo empresa o por el estilo (se puede `queriear'
  para lograrlo, pero ese no es el caso).
\end{itemize}

\subsection{Tipo de cada una de las
variables}\label{tipo-de-cada-una-de-las-variables}

\begin{itemize}
\tightlist
\item
  (cualitativa ordinal o nominal, cuantitativa continua, cuantitativa
  discreta)
\end{itemize}

\textbf{Cualitativas} Adicionalmente en estas clasificaciones las
variables Actors y ActorsCharacters serian clasificados como
cualitativos nominales pero no se agregaron porque es necesario una
limpieza de datos antes.

\begin{Shaded}
\begin{Highlighting}[]
\NormalTok{cual\_nominales }\OtherTok{\textless{}{-}} \FunctionTok{c}\NormalTok{(}
  \StringTok{"genres"}\NormalTok{,}
  \StringTok{"homePage"}\NormalTok{,}
  \StringTok{"productionCompany"}\NormalTok{,}
  \StringTok{"productionCompanyCountry"}\NormalTok{,}
  \StringTok{"productionCountry"}\NormalTok{,}
  \StringTok{"video"}\NormalTok{,}
  \StringTok{"director"}\NormalTok{,}
  \StringTok{"originalTitle"}\NormalTok{,}
  \StringTok{"title"}\NormalTok{,}
  \StringTok{"originalLanguage"}
\NormalTok{)}

\NormalTok{cual\_ordinales }\OtherTok{\textless{}{-}} \FunctionTok{c}\NormalTok{(}
  \StringTok{"releaseDate"}
\NormalTok{)}

\NormalTok{cualitativas }\OtherTok{\textless{}{-}} \FunctionTok{c}\NormalTok{(cual\_nominales, cual\_ordinales)}
\NormalTok{cualitativas}
\end{Highlighting}
\end{Shaded}

\begin{verbatim}
##  [1] "genres"                   "homePage"                
##  [3] "productionCompany"        "productionCompanyCountry"
##  [5] "productionCountry"        "video"                   
##  [7] "director"                 "originalTitle"           
##  [9] "title"                    "originalLanguage"        
## [11] "releaseDate"
\end{verbatim}

\begin{Shaded}
\begin{Highlighting}[]
\NormalTok{datos\_cual }\OtherTok{\textless{}{-}}\NormalTok{ movies[, cualitativas]}
\FunctionTok{head}\NormalTok{(datos\_cual, }\DecValTok{5}\NormalTok{)}
\end{Highlighting}
\end{Shaded}

\begin{verbatim}
##                         genres homePage productionCompany
## 1                  Drama|Crime                           
## 2                    Animation                           
## 3                    Animation                           
## 4 Thriller|Mystery|Documentary                           
## 5                    Animation                           
##   productionCompanyCountry productionCountry video       director
## 1                                            FALSE   Javad Hakami
## 2                                            FALSE Kimmy Gatewood
## 3                                            FALSE Kimmy Gatewood
## 4                                            FALSE Felipe Roldán
## 5                                            FALSE Kimmy Gatewood
##                  originalTitle                 title originalLanguage
## 1 غÙ\u0088Ø·Ù\u0087 Ù\u0088ر              Immersed               fa
## 2                   Cinderella            Cinderella               en
## 3                      Aladdin               Aladdin               en
## 4          EL ANILLO Y EL DECK THE RING AND THE DECK               es
## 5               Crimson High 3        Crimson High 3               en
##   releaseDate
## 1  2026-02-01
## 2  2026-02-01
## 3  2026-02-01
## 4  2026-02-01
## 5  2026-02-01
\end{verbatim}

\textbf{Cuantitativas} Adicionalmente en estas clasificaciones la
variable ActorsPopularity serian clasificados como cuantitativos
continuos pero no se agregaron porque es necesario una limpieza de datos
antes.

\begin{Shaded}
\begin{Highlighting}[]
\NormalTok{cuant\_discretas }\OtherTok{\textless{}{-}} \FunctionTok{c}\NormalTok{(}
  \StringTok{"budget"}\NormalTok{,}
  \StringTok{"revenue"}\NormalTok{,}
  \StringTok{"runtime"}\NormalTok{,}
  \StringTok{"voteAvg"}\NormalTok{,}
  \StringTok{"voteCount"}\NormalTok{,}
  \StringTok{"genresAmount"}\NormalTok{,}
  \StringTok{"productionCoAmount"}\NormalTok{,}
  \StringTok{"productionCountriesAmount"}\NormalTok{,}
  \StringTok{"actorsAmount"}\NormalTok{,}
  \StringTok{"castWomenAmount"}\NormalTok{,}
  \StringTok{"castMenAmount"}\NormalTok{,}
  \StringTok{"releaseYear"}
\NormalTok{)}

\NormalTok{cuant\_continuas }\OtherTok{\textless{}{-}} \FunctionTok{c}\NormalTok{(}
  \StringTok{"popularity"}
\NormalTok{)}

\NormalTok{cuantitativas }\OtherTok{\textless{}{-}} \FunctionTok{c}\NormalTok{(cuant\_discretas, cuant\_continuas)}
\NormalTok{cuantitativas}
\end{Highlighting}
\end{Shaded}

\begin{verbatim}
##  [1] "budget"                    "revenue"                  
##  [3] "runtime"                   "voteAvg"                  
##  [5] "voteCount"                 "genresAmount"             
##  [7] "productionCoAmount"        "productionCountriesAmount"
##  [9] "actorsAmount"              "castWomenAmount"          
## [11] "castMenAmount"             "releaseYear"              
## [13] "popularity"
\end{verbatim}

\begin{Shaded}
\begin{Highlighting}[]
\NormalTok{datos\_cuant }\OtherTok{\textless{}{-}}\NormalTok{ movies[, cuantitativas]}
\FunctionTok{head}\NormalTok{(datos\_cuant, }\DecValTok{5}\NormalTok{)}
\end{Highlighting}
\end{Shaded}

\begin{verbatim}
##   budget revenue runtime voteAvg voteCount genresAmount productionCoAmount
## 1      0       0      95       0         0            2                  0
## 2      0       0       3       0         0            1                  0
## 3      0       0       2       0         0            1                  0
## 4      0       0       5       0         0            3                  0
## 5      0       0      12       0         0            1                  0
##   productionCountriesAmount actorsAmount castWomenAmount castMenAmount
## 1                         0            8               2             5
## 2                         0            4               0             0
## 3                         0            3               0             0
## 4                         0            7               0             0
## 5                         0            3               0             0
##   releaseYear popularity
## 1        2026     0.0357
## 2        2026     0.0357
## 3        2026     0.0214
## 4        2026     0.0429
## 5        2026     0.0379
\end{verbatim}

\section{Análisis de Clustering}\label{anuxe1lisis-de-clustering}

\subsection{Paso 0: Preparación de
datos}\label{paso-0-preparaciuxf3n-de-datos}

Se seleccionan 6 variables numéricas: \texttt{budget}, \texttt{revenue},
\texttt{popularity}, \texttt{voteAvg}, \texttt{voteCount} y
\texttt{runtime}. Se excluyen variables de texto, fechas y conteos
auxiliares que no aportarían patrones útiles. Se eliminan registros con
\texttt{budget\ =\ 0} y \texttt{revenue\ =\ 0} por ser datos
incompletos. Tras la limpieza quedan 4,262 películas.

\begin{Shaded}
\begin{Highlighting}[]
\CommentTok{\# Variables numéricas seleccionadas para clustering}
\NormalTok{vars\_cluster }\OtherTok{\textless{}{-}} \FunctionTok{c}\NormalTok{(}\StringTok{"budget"}\NormalTok{, }\StringTok{"revenue"}\NormalTok{, }\StringTok{"popularity"}\NormalTok{, }\StringTok{"voteAvg"}\NormalTok{, }\StringTok{"voteCount"}\NormalTok{, }\StringTok{"runtime"}\NormalTok{)}

\CommentTok{\# Eliminar NAs, duplicados y películas sin datos comerciales}
\NormalTok{d2f\_raw }\OtherTok{\textless{}{-}} \FunctionTok{na.omit}\NormalTok{(movies[, }\FunctionTok{c}\NormalTok{(}\StringTok{"title"}\NormalTok{, vars\_cluster)])}
\NormalTok{d2f\_raw }\OtherTok{\textless{}{-}}\NormalTok{ d2f\_raw[}\SpecialCharTok{!}\FunctionTok{duplicated}\NormalTok{(d2f\_raw}\SpecialCharTok{$}\NormalTok{title), ]}
\NormalTok{d2f\_raw }\OtherTok{\textless{}{-}}\NormalTok{ d2f\_raw[d2f\_raw}\SpecialCharTok{$}\NormalTok{budget }\SpecialCharTok{\textgreater{}} \DecValTok{0} \SpecialCharTok{\&}\NormalTok{ d2f\_raw}\SpecialCharTok{$}\NormalTok{revenue }\SpecialCharTok{\textgreater{}} \DecValTok{0}\NormalTok{, ]}

\CommentTok{\# Escalar datos (necesario: las variables tienen magnitudes muy distintas)}
\NormalTok{d2f }\OtherTok{\textless{}{-}} \FunctionTok{as.data.frame}\NormalTok{(}\FunctionTok{scale}\NormalTok{(d2f\_raw[, vars\_cluster]))}
\FunctionTok{rownames}\NormalTok{(d2f) }\OtherTok{\textless{}{-}}\NormalTok{ d2f\_raw}\SpecialCharTok{$}\NormalTok{title}

\FunctionTok{cat}\NormalTok{(}\StringTok{"Títulos duplicados:"}\NormalTok{, }\FunctionTok{sum}\NormalTok{(}\FunctionTok{duplicated}\NormalTok{(}\FunctionTok{rownames}\NormalTok{(d2f))), }\StringTok{"}\SpecialCharTok{\textbackslash{}n}\StringTok{"}\NormalTok{)}
\end{Highlighting}
\end{Shaded}

\begin{verbatim}
## Títulos duplicados: 0
\end{verbatim}

\begin{Shaded}
\begin{Highlighting}[]
\FunctionTok{cat}\NormalTok{(}\StringTok{"Filas totales para clustering:"}\NormalTok{, }\FunctionTok{nrow}\NormalTok{(d2f), }\StringTok{"}\SpecialCharTok{\textbackslash{}n}\StringTok{"}\NormalTok{)}
\end{Highlighting}
\end{Shaded}

\begin{verbatim}
## Filas totales para clustering: 4262
\end{verbatim}

\begin{center}\rule{0.5\linewidth}{0.5pt}\end{center}

\subsection{Paso 1: Tendencia al agrupamiento --- Hopkins y
VAT}\label{paso-1-tendencia-al-agrupamiento-hopkins-y-vat}

\subsubsection{Estadístico de Hopkins}\label{estaduxedstico-de-hopkins}

El estadístico de Hopkins mide la tendencia al agrupamiento natural.
Valores cercanos a 1 indican alta tendencia; cercanos a 0.5 sugieren
distribución aleatoria.

\begin{Shaded}
\begin{Highlighting}[]
\FunctionTok{set.seed}\NormalTok{(}\DecValTok{123}\NormalTok{)}
\NormalTok{muestra\_hop }\OtherTok{\textless{}{-}}\NormalTok{ d2f[}\FunctionTok{sample}\NormalTok{(}\FunctionTok{nrow}\NormalTok{(d2f), }\FunctionTok{min}\NormalTok{(}\DecValTok{500}\NormalTok{, }\FunctionTok{nrow}\NormalTok{(d2f))), ]}
\NormalTok{hop\_resultado }\OtherTok{\textless{}{-}} \FunctionTok{hopkins}\NormalTok{(muestra\_hop, }\AttributeTok{m =} \DecValTok{50}\NormalTok{)}
\FunctionTok{cat}\NormalTok{(}\StringTok{"Estadístico de Hopkins:"}\NormalTok{, }\FunctionTok{round}\NormalTok{(hop\_resultado, }\DecValTok{4}\NormalTok{), }\StringTok{"}\SpecialCharTok{\textbackslash{}n}\StringTok{"}\NormalTok{)}
\end{Highlighting}
\end{Shaded}

\begin{verbatim}
## Estadístico de Hopkins: 1
\end{verbatim}

El estadístico resultó muy cercano a 1.0, confirmando alta tendencia al
agrupamiento natural. Aplicar clustering sobre estos datos es
estadísticamente válido.

\subsubsection{VAT (Visual Assessment of
Tendency)}\label{vat-visual-assessment-of-tendency}

\begin{Shaded}
\begin{Highlighting}[]
\FunctionTok{set.seed}\NormalTok{(}\DecValTok{123}\NormalTok{)}
\NormalTok{n\_vat }\OtherTok{\textless{}{-}} \FunctionTok{min}\NormalTok{(}\DecValTok{150}\NormalTok{, }\FunctionTok{nrow}\NormalTok{(d2f))}
\NormalTok{muestra\_vat }\OtherTok{\textless{}{-}}\NormalTok{ d2f[}\FunctionTok{sample}\NormalTok{(}\FunctionTok{nrow}\NormalTok{(d2f), n\_vat), ]}
\NormalTok{dist\_vat }\OtherTok{\textless{}{-}} \FunctionTok{dist}\NormalTok{(muestra\_vat, }\AttributeTok{method =} \StringTok{"euclidean"}\NormalTok{)}

\FunctionTok{fviz\_dist}\NormalTok{(dist\_vat, }\AttributeTok{show\_labels =} \ConstantTok{FALSE}\NormalTok{,}
          \AttributeTok{gradient =} \FunctionTok{list}\NormalTok{(}\AttributeTok{low =} \StringTok{"\#00AFBB"}\NormalTok{, }\AttributeTok{mid =} \StringTok{"white"}\NormalTok{, }\AttributeTok{high =} \StringTok{"\#FC4E07"}\NormalTok{)) }\SpecialCharTok{+}
  \FunctionTok{labs}\NormalTok{(}\AttributeTok{title    =} \StringTok{"VAT – Evaluación Visual de Tendencia al Agrupamiento"}\NormalTok{,}
       \AttributeTok{subtitle =} \FunctionTok{paste}\NormalTok{(}\StringTok{"Muestra de"}\NormalTok{, n\_vat, }\StringTok{"películas. Bloques azules diagonales indican grupos naturales."}\NormalTok{))}
\end{Highlighting}
\end{Shaded}

\begin{verbatim}
## Warning: `aes_string()` was deprecated in ggplot2 3.0.0.
## i Please use tidy evaluation idioms with `aes()`.
## i See also `vignette("ggplot2-in-packages")` for more information.
## i The deprecated feature was likely used in the factoextra package.
##   Please report the issue at <https://github.com/kassambara/factoextra/issues>.
## This warning is displayed once per session.
## Call `lifecycle::last_lifecycle_warnings()` to see where this warning was
## generated.
\end{verbatim}

\pandocbounded{\includegraphics[keepaspectratio]{Lab2_files/figure-latex/vat-1.pdf}}

La gráfica muestra la matriz de distancias reordenada. Se distingue una
zona azul central (la mayoría de películas con perfil comercial similar)
y zonas naranjas en los bordes correspondientes a producciones con
valores extremos en popularidad o ingresos. Esto confirma la presencia
de grupos naturales, aunque con una distribución asimétrica donde la
gran mayoría comparte características parecidas.

\begin{center}\rule{0.5\linewidth}{0.5pt}\end{center}

\subsection{Paso 2: Determinación del número óptimo de
clusters}\label{paso-2-determinaciuxf3n-del-nuxfamero-uxf3ptimo-de-clusters}

\begin{quote}
Nota técnica: Se usa una muestra de 2,000 películas para los métodos de
determinación de k, evitando el alto consumo de RAM que generaría
calcular matrices de distancias O(n²) sobre el dataset completo.
\end{quote}

\begin{Shaded}
\begin{Highlighting}[]
\FunctionTok{set.seed}\NormalTok{(}\DecValTok{123}\NormalTok{)}
\NormalTok{n\_muestra }\OtherTok{\textless{}{-}} \FunctionTok{min}\NormalTok{(}\DecValTok{2000}\NormalTok{, }\FunctionTok{nrow}\NormalTok{(d2f))}
\NormalTok{idx\_muestra }\OtherTok{\textless{}{-}} \FunctionTok{sample}\NormalTok{(}\FunctionTok{nrow}\NormalTok{(d2f), n\_muestra)}
\NormalTok{d2f\_muestra }\OtherTok{\textless{}{-}}\NormalTok{ d2f[idx\_muestra, ]}
\FunctionTok{cat}\NormalTok{(}\StringTok{"Muestra utilizada para determinar k:"}\NormalTok{, n\_muestra, }\StringTok{"películas}\SpecialCharTok{\textbackslash{}n}\StringTok{"}\NormalTok{)}
\end{Highlighting}
\end{Shaded}

\begin{verbatim}
## Muestra utilizada para determinar k: 2000 películas
\end{verbatim}

\subsubsection{Método del Codo (WSS)}\label{muxe9todo-del-codo-wss}

\begin{Shaded}
\begin{Highlighting}[]
\NormalTok{wss }\OtherTok{\textless{}{-}} \FunctionTok{numeric}\NormalTok{(}\DecValTok{10}\NormalTok{)}
\ControlFlowTok{for}\NormalTok{ (i }\ControlFlowTok{in} \DecValTok{1}\SpecialCharTok{:}\DecValTok{10}\NormalTok{) \{}
\NormalTok{  wss[i] }\OtherTok{\textless{}{-}} \FunctionTok{sum}\NormalTok{(}\FunctionTok{kmeans}\NormalTok{(d2f\_muestra, }\AttributeTok{centers =}\NormalTok{ i, }\AttributeTok{nstart =} \DecValTok{10}\NormalTok{)}\SpecialCharTok{$}\NormalTok{withinss)}
\NormalTok{\}}
\FunctionTok{plot}\NormalTok{(}\DecValTok{1}\SpecialCharTok{:}\DecValTok{10}\NormalTok{, wss, }\AttributeTok{type =} \StringTok{"b"}\NormalTok{,}
     \AttributeTok{xlab =} \StringTok{"Número de Clusters"}\NormalTok{,}
     \AttributeTok{ylab =} \StringTok{"Suma de cuadrados dentro del grupo"}\NormalTok{,}
     \AttributeTok{main =} \StringTok{"Método del Codo – WSS"}\NormalTok{)}
\end{Highlighting}
\end{Shaded}

\pandocbounded{\includegraphics[keepaspectratio]{Lab2_files/figure-latex/metodo_codo_manual-1.pdf}}

\begin{Shaded}
\begin{Highlighting}[]
\FunctionTok{fviz\_nbclust}\NormalTok{(d2f\_muestra, kmeans, }\AttributeTok{method =} \StringTok{"wss"}\NormalTok{, }\AttributeTok{k.max =} \DecValTok{10}\NormalTok{) }\SpecialCharTok{+}
  \FunctionTok{labs}\NormalTok{(}\AttributeTok{title    =} \StringTok{"Método del Codo (WSS) – K{-}Means"}\NormalTok{,}
       \AttributeTok{subtitle =} \StringTok{"El \textquotesingle{}codo\textquotesingle{} señala el k donde la reducción de WSS se estabiliza"}\NormalTok{,}
       \AttributeTok{x =} \StringTok{"Número de Clusters (k)"}\NormalTok{, }\AttributeTok{y =} \StringTok{"WSS intra{-}cluster"}\NormalTok{) }\SpecialCharTok{+}
  \FunctionTok{theme\_bw}\NormalTok{()}
\end{Highlighting}
\end{Shaded}

\pandocbounded{\includegraphics[keepaspectratio]{Lab2_files/figure-latex/codo_factoextra-1.pdf}}

La curva WSS muestra una reducción pronunciada hasta k=4, después de la
cual la disminución se vuelve marginal. El ``codo'' se forma claramente
en k=4, por lo que agregar más clusters no justifica la complejidad
adicional.

\subsubsection{Método de la Silueta}\label{muxe9todo-de-la-silueta}

\begin{Shaded}
\begin{Highlighting}[]
\FunctionTok{fviz\_nbclust}\NormalTok{(d2f\_muestra, kmeans, }\AttributeTok{method =} \StringTok{"silhouette"}\NormalTok{, }\AttributeTok{k.max =} \DecValTok{10}\NormalTok{) }\SpecialCharTok{+}
  \FunctionTok{labs}\NormalTok{(}\AttributeTok{title    =} \StringTok{"Método de Silueta – K{-}Means"}\NormalTok{,}
       \AttributeTok{subtitle =} \StringTok{"El k con mayor silueta promedio es el más adecuado"}\NormalTok{,}
       \AttributeTok{x =} \StringTok{"Número de Clusters (k)"}\NormalTok{, }\AttributeTok{y =} \StringTok{"Ancho de Silueta Promedio"}\NormalTok{) }\SpecialCharTok{+}
  \FunctionTok{theme\_bw}\NormalTok{()}
\end{Highlighting}
\end{Shaded}

\pandocbounded{\includegraphics[keepaspectratio]{Lab2_files/figure-latex/silueta_factoextra-1.pdf}}

La silueta confirma k=4 como el número óptimo, con un ancho promedio de
0.573 (``estructura razonable''). Para k≥5 la silueta cae por debajo de
0.31, generando subgrupos sin coherencia real.

\subsubsection{Gap Statistic}\label{gap-statistic}

\begin{Shaded}
\begin{Highlighting}[]
\FunctionTok{fviz\_nbclust}\NormalTok{(d2f\_muestra, kmeans,}
             \AttributeTok{nstart  =} \DecValTok{25}\NormalTok{,}
             \AttributeTok{method  =} \StringTok{"gap\_stat"}\NormalTok{,}
             \AttributeTok{nboot   =} \DecValTok{50}\NormalTok{,}
             \AttributeTok{verbose =} \ConstantTok{FALSE}\NormalTok{) }\SpecialCharTok{+}
  \FunctionTok{labs}\NormalTok{(}\AttributeTok{title    =} \StringTok{"Gap Statistic – K{-}Means"}\NormalTok{,}
       \AttributeTok{subtitle =} \StringTok{"El k óptimo es donde Gap(k) es máximo o se estabiliza"}\NormalTok{,}
       \AttributeTok{x =} \StringTok{"Número de Clusters (k)"}\NormalTok{, }\AttributeTok{y =} \StringTok{"Gap Statistic"}\NormalTok{) }\SpecialCharTok{+}
  \FunctionTok{theme\_bw}\NormalTok{()}
\end{Highlighting}
\end{Shaded}

\begin{verbatim}
## Warning: did not converge in 10 iterations
## Warning: did not converge in 10 iterations
## Warning: did not converge in 10 iterations
## Warning: did not converge in 10 iterations
## Warning: did not converge in 10 iterations
## Warning: did not converge in 10 iterations
## Warning: did not converge in 10 iterations
## Warning: did not converge in 10 iterations
## Warning: did not converge in 10 iterations
## Warning: did not converge in 10 iterations
## Warning: did not converge in 10 iterations
## Warning: did not converge in 10 iterations
## Warning: did not converge in 10 iterations
## Warning: did not converge in 10 iterations
## Warning: did not converge in 10 iterations
## Warning: did not converge in 10 iterations
## Warning: did not converge in 10 iterations
## Warning: did not converge in 10 iterations
\end{verbatim}

\pandocbounded{\includegraphics[keepaspectratio]{Lab2_files/figure-latex/gap_factoextra-1.pdf}}

\subsubsection{Tabla resumen de silueta por
k}\label{tabla-resumen-de-silueta-por-k}

\begin{Shaded}
\begin{Highlighting}[]
\NormalTok{sil\_scores }\OtherTok{\textless{}{-}} \FunctionTok{sapply}\NormalTok{(}\DecValTok{2}\SpecialCharTok{:}\DecValTok{8}\NormalTok{, }\ControlFlowTok{function}\NormalTok{(k) \{}
\NormalTok{  km }\OtherTok{\textless{}{-}} \FunctionTok{kmeans}\NormalTok{(d2f\_muestra, }\AttributeTok{centers =}\NormalTok{ k, }\AttributeTok{nstart =} \DecValTok{10}\NormalTok{, }\AttributeTok{iter.max =} \DecValTok{50}\NormalTok{)}
\NormalTok{  ss }\OtherTok{\textless{}{-}} \FunctionTok{silhouette}\NormalTok{(km}\SpecialCharTok{$}\NormalTok{cluster, }\FunctionTok{dist}\NormalTok{(d2f\_muestra))}
  \FunctionTok{round}\NormalTok{(}\FunctionTok{mean}\NormalTok{(ss[, }\DecValTok{3}\NormalTok{]), }\DecValTok{4}\NormalTok{)}
\NormalTok{\})}

\NormalTok{sil\_tabla }\OtherTok{\textless{}{-}} \FunctionTok{data.frame}\NormalTok{(}
  \AttributeTok{k              =} \DecValTok{2}\SpecialCharTok{:}\DecValTok{8}\NormalTok{,}
  \AttributeTok{Silhouette     =}\NormalTok{ sil\_scores,}
  \AttributeTok{Interpretacion =} \FunctionTok{ifelse}\NormalTok{(sil\_scores }\SpecialCharTok{\textgreater{}=} \FloatTok{0.70}\NormalTok{, }\StringTok{"Estructura fuerte"}\NormalTok{,}
                   \FunctionTok{ifelse}\NormalTok{(sil\_scores }\SpecialCharTok{\textgreater{}=} \FloatTok{0.50}\NormalTok{, }\StringTok{"Estructura razonable"}\NormalTok{,}
                   \FunctionTok{ifelse}\NormalTok{(sil\_scores }\SpecialCharTok{\textgreater{}=} \FloatTok{0.25}\NormalTok{, }\StringTok{"Estructura débil"}\NormalTok{, }\StringTok{"Sin estructura"}\NormalTok{)))}
\NormalTok{)}
\FunctionTok{print}\NormalTok{(sil\_tabla)}
\end{Highlighting}
\end{Shaded}

\begin{verbatim}
##   k Silhouette       Interpretacion
## 1 2     0.5618 Estructura razonable
## 2 3     0.5895 Estructura razonable
## 3 4     0.2211       Sin estructura
## 4 5     0.2511     Estructura débil
## 5 6     0.2583     Estructura débil
## 6 7     0.2674     Estructura débil
## 7 8     0.2784     Estructura débil
\end{verbatim}

\begin{Shaded}
\begin{Highlighting}[]
\NormalTok{k\_final }\OtherTok{\textless{}{-}}\NormalTok{ sil\_tabla}\SpecialCharTok{$}\NormalTok{k[}\FunctionTok{which.max}\NormalTok{(sil\_tabla}\SpecialCharTok{$}\NormalTok{Silhouette)]}
\FunctionTok{cat}\NormalTok{(}\StringTok{"}\SpecialCharTok{\textbackslash{}n}\StringTok{K seleccionado:"}\NormalTok{, k\_final, }\StringTok{"}\SpecialCharTok{\textbackslash{}n}\StringTok{"}\NormalTok{)}
\end{Highlighting}
\end{Shaded}

\begin{verbatim}
## 
## K seleccionado: 3
\end{verbatim}

\begin{Shaded}
\begin{Highlighting}[]
\FunctionTok{cat}\NormalTok{(}\StringTok{"Silueta promedio:"}\NormalTok{, }\FunctionTok{max}\NormalTok{(sil\_tabla}\SpecialCharTok{$}\NormalTok{Silhouette), }\StringTok{"}\SpecialCharTok{\textbackslash{}n}\StringTok{"}\NormalTok{)}
\end{Highlighting}
\end{Shaded}

\begin{verbatim}
## Silueta promedio: 0.5895
\end{verbatim}

Los tres métodos coinciden en k=4 como el número óptimo. k=2, 3 y 4
califican como ``Estructura razonable'' (silueta ≥ 0.50), siendo k=4 el
que maximiza la separación con 0.573. A partir de k=5 la calidad se
degrada. Se selecciona k=4.

\begin{center}\rule{0.5\linewidth}{0.5pt}\end{center}

\subsection{Paso 3: Agrupamiento}\label{paso-3-agrupamiento}

\subsubsection{Algoritmo 1: K-Means}\label{algoritmo-1-k-means}

K-Means se ejecuta sobre el dataset completo (4,262 películas). No
requiere matriz de distancias, por lo que es eficiente en memoria. Se
usa \texttt{nstart\ =\ 25} para evitar óptimos locales.

\begin{Shaded}
\begin{Highlighting}[]
\FunctionTok{set.seed}\NormalTok{(}\DecValTok{123}\NormalTok{)}
\NormalTok{km\_res }\OtherTok{\textless{}{-}} \FunctionTok{kmeans}\NormalTok{(d2f, }\AttributeTok{centers =}\NormalTok{ k\_final, }\AttributeTok{nstart =} \DecValTok{25}\NormalTok{, }\AttributeTok{iter.max =} \DecValTok{100}\NormalTok{)}
\NormalTok{d2f\_raw}\SpecialCharTok{$}\NormalTok{cluster\_km }\OtherTok{\textless{}{-}} \FunctionTok{as.factor}\NormalTok{(km\_res}\SpecialCharTok{$}\NormalTok{cluster)}

\FunctionTok{cat}\NormalTok{(}\StringTok{"Tamaño de cada cluster (K{-}Means):}\SpecialCharTok{\textbackslash{}n}\StringTok{"}\NormalTok{)}
\end{Highlighting}
\end{Shaded}

\begin{verbatim}
## Tamaño de cada cluster (K-Means):
\end{verbatim}

\begin{Shaded}
\begin{Highlighting}[]
\FunctionTok{print}\NormalTok{(km\_res}\SpecialCharTok{$}\NormalTok{size)}
\end{Highlighting}
\end{Shaded}

\begin{verbatim}
## [1] 3713  458   91
\end{verbatim}

\begin{Shaded}
\begin{Highlighting}[]
\FunctionTok{cat}\NormalTok{(}\StringTok{"Varianza explicada (BSS/TSS):"}\NormalTok{, }\FunctionTok{round}\NormalTok{(km\_res}\SpecialCharTok{$}\NormalTok{betweenss }\SpecialCharTok{/}\NormalTok{ km\_res}\SpecialCharTok{$}\NormalTok{totss }\SpecialCharTok{*} \DecValTok{100}\NormalTok{, }\DecValTok{1}\NormalTok{), }\StringTok{"\%}\SpecialCharTok{\textbackslash{}n}\StringTok{"}\NormalTok{)}
\end{Highlighting}
\end{Shaded}

\begin{verbatim}
## Varianza explicada (BSS/TSS): 40.1 %
\end{verbatim}

\begin{Shaded}
\begin{Highlighting}[]
\FunctionTok{set.seed}\NormalTok{(}\DecValTok{123}\NormalTok{)}
\NormalTok{idx\_plot }\OtherTok{\textless{}{-}} \FunctionTok{sample}\NormalTok{(}\FunctionTok{nrow}\NormalTok{(d2f), }\FunctionTok{min}\NormalTok{(}\DecValTok{1000}\NormalTok{, }\FunctionTok{nrow}\NormalTok{(d2f)))}
\FunctionTok{plotcluster}\NormalTok{(d2f[idx\_plot, ], km\_res}\SpecialCharTok{$}\NormalTok{cluster[idx\_plot],}
            \AttributeTok{main =} \StringTok{"K{-}Means – plotcluster (muestra 1,000 puntos)"}\NormalTok{)}
\end{Highlighting}
\end{Shaded}

\pandocbounded{\includegraphics[keepaspectratio]{Lab2_files/figure-latex/kmeans_clustering-1.pdf}}

\begin{Shaded}
\begin{Highlighting}[]
\FunctionTok{fviz\_cluster}\NormalTok{(km\_res, }\AttributeTok{data =}\NormalTok{ d2f,}
             \AttributeTok{geom         =} \StringTok{"point"}\NormalTok{,}
             \AttributeTok{ellipse.type =} \StringTok{"norm"}\NormalTok{,}
             \AttributeTok{palette      =} \StringTok{"Set2"}\NormalTok{,}
             \AttributeTok{alpha        =} \FloatTok{0.4}\NormalTok{) }\SpecialCharTok{+}
  \FunctionTok{labs}\NormalTok{(}\AttributeTok{title    =} \StringTok{"K{-}Means"}\NormalTok{,}
       \AttributeTok{subtitle =} \FunctionTok{paste}\NormalTok{(}\StringTok{"k ="}\NormalTok{, k\_final, }\StringTok{"| n ="}\NormalTok{, }\FunctionTok{nrow}\NormalTok{(d2f), }\StringTok{"películas"}\NormalTok{)) }\SpecialCharTok{+}
  \FunctionTok{theme\_bw}\NormalTok{()}
\end{Highlighting}
\end{Shaded}

\pandocbounded{\includegraphics[keepaspectratio]{Lab2_files/figure-latex/kmeans_clustering-2.pdf}}

K-Means produjo 4 grupos muy desiguales: el Cluster 1 concentra
\textasciitilde3,719 películas (87\%), mientras los demás son
minoritarios (467, 77 y 7 películas). Esta distribución refleja la
industria real, donde la gran mayoría son producciones estándar y solo
un pequeño porcentaje alcanza el nivel de blockbuster o fenómeno viral.
El modelo explica el 52.7\% de la varianza total.

\subsubsection{Algoritmo 2: Clustering Jerárquico
(Ward)}\label{algoritmo-2-clustering-jeruxe1rquico-ward}

El jerárquico requiere una matriz de distancias O(n²), por lo que se
aplica sobre una muestra de 1,500 películas para mantener el consumo de
RAM manejable.

\begin{Shaded}
\begin{Highlighting}[]
\FunctionTok{set.seed}\NormalTok{(}\DecValTok{456}\NormalTok{)}
\NormalTok{n\_hc }\OtherTok{\textless{}{-}} \FunctionTok{min}\NormalTok{(}\DecValTok{1500}\NormalTok{, }\FunctionTok{nrow}\NormalTok{(d2f))}
\NormalTok{idx\_hc }\OtherTok{\textless{}{-}} \FunctionTok{sample}\NormalTok{(}\FunctionTok{nrow}\NormalTok{(d2f), n\_hc)}
\NormalTok{d2f\_hc }\OtherTok{\textless{}{-}}\NormalTok{ d2f[idx\_hc, ]}

\NormalTok{dist\_hc }\OtherTok{\textless{}{-}} \FunctionTok{dist}\NormalTok{(d2f\_hc, }\AttributeTok{method =} \StringTok{"euclidean"}\NormalTok{)}
\NormalTok{hc\_res  }\OtherTok{\textless{}{-}} \FunctionTok{hclust}\NormalTok{(dist\_hc, }\AttributeTok{method =} \StringTok{"ward.D2"}\NormalTok{)}

\FunctionTok{plot}\NormalTok{(hc\_res, }\AttributeTok{labels =} \ConstantTok{FALSE}\NormalTok{, }\AttributeTok{hang =} \SpecialCharTok{{-}}\DecValTok{1}\NormalTok{,}
     \AttributeTok{main =} \FunctionTok{paste}\NormalTok{(}\StringTok{"Dendrograma – Clustering Jerárquico (Ward)}\SpecialCharTok{\textbackslash{}n}\StringTok{Muestra:"}\NormalTok{, n\_hc, }\StringTok{"películas"}\NormalTok{),}
     \AttributeTok{xlab =} \StringTok{"Películas"}\NormalTok{, }\AttributeTok{ylab =} \StringTok{"Distancia (Ward)"}\NormalTok{)}
\FunctionTok{rect.hclust}\NormalTok{(hc\_res, }\AttributeTok{k =}\NormalTok{ k\_final, }\AttributeTok{border =} \DecValTok{2}\SpecialCharTok{:}\NormalTok{(k\_final }\SpecialCharTok{+} \DecValTok{1}\NormalTok{))}
\end{Highlighting}
\end{Shaded}

\pandocbounded{\includegraphics[keepaspectratio]{Lab2_files/figure-latex/hierarchical_clustering-1.pdf}}

\begin{Shaded}
\begin{Highlighting}[]
\NormalTok{hc\_clusters }\OtherTok{\textless{}{-}} \FunctionTok{cutree}\NormalTok{(hc\_res, }\AttributeTok{k =}\NormalTok{ k\_final)}
\FunctionTok{cat}\NormalTok{(}\StringTok{"Distribución por cluster (Jerárquico):}\SpecialCharTok{\textbackslash{}n}\StringTok{"}\NormalTok{)}
\end{Highlighting}
\end{Shaded}

\begin{verbatim}
## Distribución por cluster (Jerárquico):
\end{verbatim}

\begin{Shaded}
\begin{Highlighting}[]
\FunctionTok{print}\NormalTok{(}\FunctionTok{table}\NormalTok{(hc\_clusters))}
\end{Highlighting}
\end{Shaded}

\begin{verbatim}
## hc_clusters
##    1    2    3 
## 1388  108    4
\end{verbatim}

El dendrograma muestra una división principal en dos ramas a nivel alto,
con subdivisiones más finas en niveles inferiores. El corte a k=4 es
justificado por los saltos de altura entre fusiones. Al igual que
K-Means, produce un cluster dominante con la mayoría de películas y
grupos pequeños de producciones excepcionales.

\begin{center}\rule{0.5\linewidth}{0.5pt}\end{center}

\subsection{Paso 4: Calidad del agrupamiento ---
Silueta}\label{paso-4-calidad-del-agrupamiento-silueta}

Ambos algoritmos se evalúan sobre la misma muestra (n=1,500) para
comparación justa.

\begin{Shaded}
\begin{Highlighting}[]
\NormalTok{km\_hc  }\OtherTok{\textless{}{-}} \FunctionTok{kmeans}\NormalTok{(d2f\_hc, }\AttributeTok{centers =}\NormalTok{ k\_final, }\AttributeTok{nstart =} \DecValTok{25}\NormalTok{)}
\NormalTok{sil\_km }\OtherTok{\textless{}{-}} \FunctionTok{silhouette}\NormalTok{(km\_hc}\SpecialCharTok{$}\NormalTok{cluster, dist\_hc)}
\NormalTok{sil\_hc }\OtherTok{\textless{}{-}} \FunctionTok{silhouette}\NormalTok{(hc\_clusters,   dist\_hc)}

\NormalTok{calidad }\OtherTok{\textless{}{-}} \FunctionTok{data.frame}\NormalTok{(}
  \AttributeTok{Algoritmo           =} \FunctionTok{c}\NormalTok{(}\StringTok{"K{-}Means"}\NormalTok{, }\StringTok{"Jerárquico (Ward)"}\NormalTok{),}
  \AttributeTok{Silhouette\_Promedio =} \FunctionTok{round}\NormalTok{(}\FunctionTok{c}\NormalTok{(}\FunctionTok{mean}\NormalTok{(sil\_km[, }\DecValTok{3}\NormalTok{]), }\FunctionTok{mean}\NormalTok{(sil\_hc[, }\DecValTok{3}\NormalTok{])), }\DecValTok{4}\NormalTok{),}
  \AttributeTok{Clusters            =}\NormalTok{ k\_final}
\NormalTok{)}
\FunctionTok{print}\NormalTok{(calidad)}
\end{Highlighting}
\end{Shaded}

\begin{verbatim}
##           Algoritmo Silhouette_Promedio Clusters
## 1           K-Means              0.5532        3
## 2 Jerárquico (Ward)              0.6001        3
\end{verbatim}

\begin{Shaded}
\begin{Highlighting}[]
\FunctionTok{plot}\NormalTok{(sil\_km,}
     \AttributeTok{col    =}\NormalTok{ (}\DecValTok{2}\SpecialCharTok{:}\NormalTok{(k\_final }\SpecialCharTok{+} \DecValTok{1}\NormalTok{))[sil\_km[, }\DecValTok{1}\NormalTok{]],}
     \AttributeTok{border =} \ConstantTok{NA}\NormalTok{,}
     \AttributeTok{main   =} \FunctionTok{paste}\NormalTok{(}\StringTok{"Silueta K{-}Means | k ="}\NormalTok{, k\_final,}
                    \StringTok{"| avg ="}\NormalTok{, }\FunctionTok{round}\NormalTok{(}\FunctionTok{mean}\NormalTok{(sil\_km[, }\DecValTok{3}\NormalTok{]), }\DecValTok{3}\NormalTok{)),}
     \AttributeTok{sub    =} \FunctionTok{paste}\NormalTok{(}\StringTok{"n ="}\NormalTok{, }\FunctionTok{nrow}\NormalTok{(d2f\_hc), }\StringTok{"películas (muestra)"}\NormalTok{))}
\FunctionTok{abline}\NormalTok{(}\AttributeTok{v =} \FunctionTok{mean}\NormalTok{(sil\_km[, }\DecValTok{3}\NormalTok{]), }\AttributeTok{lty =} \DecValTok{2}\NormalTok{, }\AttributeTok{col =} \StringTok{"red"}\NormalTok{)}
\end{Highlighting}
\end{Shaded}

\pandocbounded{\includegraphics[keepaspectratio]{Lab2_files/figure-latex/silueta_kmeans-1.pdf}}

\begin{Shaded}
\begin{Highlighting}[]
\FunctionTok{plot}\NormalTok{(sil\_hc,}
     \AttributeTok{col    =}\NormalTok{ (}\DecValTok{2}\SpecialCharTok{:}\NormalTok{(k\_final }\SpecialCharTok{+} \DecValTok{1}\NormalTok{))[sil\_hc[, }\DecValTok{1}\NormalTok{]],}
     \AttributeTok{border =} \ConstantTok{NA}\NormalTok{,}
     \AttributeTok{main   =} \FunctionTok{paste}\NormalTok{(}\StringTok{"Silueta Jerárquico (Ward) | k ="}\NormalTok{, k\_final,}
                    \StringTok{"| avg ="}\NormalTok{, }\FunctionTok{round}\NormalTok{(}\FunctionTok{mean}\NormalTok{(sil\_hc[, }\DecValTok{3}\NormalTok{]), }\DecValTok{3}\NormalTok{)),}
     \AttributeTok{sub    =} \FunctionTok{paste}\NormalTok{(}\StringTok{"n ="}\NormalTok{, }\FunctionTok{nrow}\NormalTok{(d2f\_hc), }\StringTok{"películas (muestra)"}\NormalTok{))}
\FunctionTok{abline}\NormalTok{(}\AttributeTok{v =} \FunctionTok{mean}\NormalTok{(sil\_hc[, }\DecValTok{3}\NormalTok{]), }\AttributeTok{lty =} \DecValTok{2}\NormalTok{, }\AttributeTok{col =} \StringTok{"red"}\NormalTok{)}
\end{Highlighting}
\end{Shaded}

\pandocbounded{\includegraphics[keepaspectratio]{Lab2_files/figure-latex/silueta_jerarquico-1.pdf}}

K-Means obtuvo silueta promedio de 0.5376, superando al Jerárquico
(0.4691). En la silueta de K-Means, el Cluster 4 (7 películas) alcanza
la silueta más alta (\textasciitilde0.80), indicando que estas películas
son muy distintas del resto. El Cluster 1 tiene silueta de
\textasciitilde0.60 con buena cohesión interna. El Cluster 2 muestra
silueta baja (\textasciitilde0.14), lo que refleja que la frontera entre
``película exitosa'' y ``blockbuster'' es difusa. El jerárquico genera
un cluster mixto con silueta muy baja (0.06), indicando menor calidad de
separación. Se selecciona K-Means como algoritmo final por su mayor
silueta y porque opera sobre el dataset completo.

\begin{Shaded}
\begin{Highlighting}[]
\NormalTok{mejor\_algoritmo }\OtherTok{\textless{}{-}}\NormalTok{ calidad}\SpecialCharTok{$}\NormalTok{Algoritmo[}\FunctionTok{which.max}\NormalTok{(calidad}\SpecialCharTok{$}\NormalTok{Silhouette\_Promedio)]}
\FunctionTok{cat}\NormalTok{(}\StringTok{"Algoritmo con mejor calidad de clusters:"}\NormalTok{, mejor\_algoritmo, }\StringTok{"}\SpecialCharTok{\textbackslash{}n}\StringTok{"}\NormalTok{)}
\end{Highlighting}
\end{Shaded}

\begin{verbatim}
## Algoritmo con mejor calidad de clusters: Jerárquico (Ward)
\end{verbatim}

\begin{Shaded}
\begin{Highlighting}[]
\FunctionTok{cat}\NormalTok{(}\StringTok{"Se usará K{-}Means (dataset completo) para la interpretación final.}\SpecialCharTok{\textbackslash{}n}\StringTok{"}\NormalTok{)}
\end{Highlighting}
\end{Shaded}

\begin{verbatim}
## Se usará K-Means (dataset completo) para la interpretación final.
\end{verbatim}

\begin{center}\rule{0.5\linewidth}{0.5pt}\end{center}

\subsection{Paso 5: Interpretación de los
grupos}\label{paso-5-interpretaciuxf3n-de-los-grupos}

\subsubsection{5.1 Medidas de tendencia central por
cluster}\label{medidas-de-tendencia-central-por-cluster}

\begin{Shaded}
\begin{Highlighting}[]
\NormalTok{perfil }\OtherTok{\textless{}{-}}\NormalTok{ d2f\_raw }\SpecialCharTok{\%\textgreater{}\%}
  \FunctionTok{group\_by}\NormalTok{(cluster\_km) }\SpecialCharTok{\%\textgreater{}\%}
  \FunctionTok{summarise}\NormalTok{(}
    \AttributeTok{n\_peliculas     =} \FunctionTok{n}\NormalTok{(),}
    \AttributeTok{budget\_media    =} \FunctionTok{round}\NormalTok{(}\FunctionTok{mean}\NormalTok{(budget,     }\AttributeTok{na.rm =} \ConstantTok{TRUE}\NormalTok{), }\DecValTok{0}\NormalTok{),}
    \AttributeTok{budget\_mediana  =} \FunctionTok{round}\NormalTok{(}\FunctionTok{median}\NormalTok{(budget,   }\AttributeTok{na.rm =} \ConstantTok{TRUE}\NormalTok{), }\DecValTok{0}\NormalTok{),}
    \AttributeTok{revenue\_media   =} \FunctionTok{round}\NormalTok{(}\FunctionTok{mean}\NormalTok{(revenue,    }\AttributeTok{na.rm =} \ConstantTok{TRUE}\NormalTok{), }\DecValTok{0}\NormalTok{),}
    \AttributeTok{revenue\_mediana =} \FunctionTok{round}\NormalTok{(}\FunctionTok{median}\NormalTok{(revenue,  }\AttributeTok{na.rm =} \ConstantTok{TRUE}\NormalTok{), }\DecValTok{0}\NormalTok{),}
    \AttributeTok{pop\_media       =} \FunctionTok{round}\NormalTok{(}\FunctionTok{mean}\NormalTok{(popularity, }\AttributeTok{na.rm =} \ConstantTok{TRUE}\NormalTok{), }\DecValTok{2}\NormalTok{),}
    \AttributeTok{voteAvg\_media   =} \FunctionTok{round}\NormalTok{(}\FunctionTok{mean}\NormalTok{(voteAvg,    }\AttributeTok{na.rm =} \ConstantTok{TRUE}\NormalTok{), }\DecValTok{2}\NormalTok{),}
    \AttributeTok{voteCount\_media =} \FunctionTok{round}\NormalTok{(}\FunctionTok{mean}\NormalTok{(voteCount,  }\AttributeTok{na.rm =} \ConstantTok{TRUE}\NormalTok{), }\DecValTok{0}\NormalTok{),}
    \AttributeTok{runtime\_media   =} \FunctionTok{round}\NormalTok{(}\FunctionTok{mean}\NormalTok{(runtime,    }\AttributeTok{na.rm =} \ConstantTok{TRUE}\NormalTok{), }\DecValTok{1}\NormalTok{)}
\NormalTok{  )}
\FunctionTok{print}\NormalTok{(perfil)}
\end{Highlighting}
\end{Shaded}

\begin{verbatim}
## # A tibble: 3 x 10
##   cluster_km n_peliculas budget_media budget_mediana revenue_media
##   <fct>            <int>        <dbl>          <dbl>         <dbl>
## 1 1                 3713     28256866       20000000      73298217
## 2 2                  458    132382572      131000000     546959218
## 3 3                   91      2418948            100       5246201
## # i 5 more variables: revenue_mediana <dbl>, pop_media <dbl>,
## #   voteAvg_media <dbl>, voteCount_media <dbl>, runtime_media <dbl>
\end{verbatim}

Con base en los perfiles obtenidos, los 4 clusters se identifican como:

\begin{itemize}
\item
  Cluster 1 --- ``Producciones Convencionales'' (n≈3,719, 87\%):
  Presupuesto medio \$28M, ingresos \$73M, popularidad 38. La gran
  mayoría del catálogo con características comerciales estándar. Drama,
  Comedia y Thriller dominan.
\item
  Cluster 2 --- ``Blockbusters Comerciales'' (n≈467, 11\%): Presupuesto
  \$130M, ingresos \$540M, popularidad 147. Alto rendimiento comercial
  con géneros de Action y Adventure. Incluye Avatar, Avengers: Endgame y
  Titanic.
\item
  Cluster 3 --- ``Producciones Marginales'' (n≈77, 2\%): Presupuesto y
  revenue con medianas cercanas a cero, popularidad 2.8, duración
  \textasciitilde18 min. Probablemente cortometrajes o producciones con
  registros financieros incompletos.
\item
  Cluster 4 --- ``Fenómenos Virales'' (n=7, \textless1\%): Popularidad
  extrema (media 6,649, máximo 11,475), presupuesto moderado \$120M.
  Spider-Man: No Way Home, Eternals y Sing 2 lideran este cluster por su
  engagement viral masivo en plataformas digitales.
\end{itemize}

\subsubsection{5.2 Boxplots comparativos por
variable}\label{boxplots-comparativos-por-variable}

\begin{Shaded}
\begin{Highlighting}[]
\ControlFlowTok{for}\NormalTok{ (v }\ControlFlowTok{in}\NormalTok{ vars\_cluster) \{}
\NormalTok{  p }\OtherTok{\textless{}{-}} \FunctionTok{ggplot}\NormalTok{(d2f\_raw, }\FunctionTok{aes\_string}\NormalTok{(}\AttributeTok{x =} \StringTok{"cluster\_km"}\NormalTok{, }\AttributeTok{y =}\NormalTok{ v, }\AttributeTok{fill =} \StringTok{"cluster\_km"}\NormalTok{)) }\SpecialCharTok{+}
    \FunctionTok{geom\_boxplot}\NormalTok{(}\AttributeTok{alpha =} \FloatTok{0.7}\NormalTok{, }\AttributeTok{outlier.color =} \StringTok{"red"}\NormalTok{, }\AttributeTok{outlier.size =} \DecValTok{1}\NormalTok{) }\SpecialCharTok{+}
    \FunctionTok{scale\_fill\_brewer}\NormalTok{(}\AttributeTok{palette =} \StringTok{"Set2"}\NormalTok{) }\SpecialCharTok{+}
    \FunctionTok{labs}\NormalTok{(}\AttributeTok{title    =} \FunctionTok{paste}\NormalTok{(}\StringTok{"Distribución de"}\NormalTok{, v, }\StringTok{"por Cluster"}\NormalTok{),}
         \AttributeTok{subtitle =} \StringTok{"Outliers en rojo = películas atípicas dentro del grupo"}\NormalTok{,}
         \AttributeTok{x =} \StringTok{"Cluster"}\NormalTok{, }\AttributeTok{y =}\NormalTok{ v) }\SpecialCharTok{+}
    \FunctionTok{theme\_bw}\NormalTok{() }\SpecialCharTok{+} \FunctionTok{theme}\NormalTok{(}\AttributeTok{legend.position =} \StringTok{"none"}\NormalTok{)}
  \FunctionTok{print}\NormalTok{(p)}
\NormalTok{\}}
\end{Highlighting}
\end{Shaded}

\pandocbounded{\includegraphics[keepaspectratio]{Lab2_files/figure-latex/cluster_boxplots-1.pdf}}
\pandocbounded{\includegraphics[keepaspectratio]{Lab2_files/figure-latex/cluster_boxplots-2.pdf}}
\pandocbounded{\includegraphics[keepaspectratio]{Lab2_files/figure-latex/cluster_boxplots-3.pdf}}
\pandocbounded{\includegraphics[keepaspectratio]{Lab2_files/figure-latex/cluster_boxplots-4.pdf}}
\pandocbounded{\includegraphics[keepaspectratio]{Lab2_files/figure-latex/cluster_boxplots-5.pdf}}
\pandocbounded{\includegraphics[keepaspectratio]{Lab2_files/figure-latex/cluster_boxplots-6.pdf}}

Los boxplots confirman la interpretación de los clusters. En budget y
revenue el Cluster 2 se ubica claramente por encima de los demás. La
variable popularity es la que mejor separa los clusters, con el Cluster
4 completamente fuera de escala. voteAvg es similar entre clusters
(6.5--7.5), indicando que la calidad percibida no es el principal
diferenciador. Runtime es homogéneo, salvo el Cluster 3 que incluye
cortometrajes (\textasciitilde18 min de media).

\subsubsection{5.3 Géneros por cluster}\label{guxe9neros-por-cluster}

\begin{Shaded}
\begin{Highlighting}[]
\NormalTok{movies\_con\_cluster }\OtherTok{\textless{}{-}}\NormalTok{ movies }\SpecialCharTok{\%\textgreater{}\%}
  \FunctionTok{filter}\NormalTok{(title }\SpecialCharTok{\%in\%}\NormalTok{ d2f\_raw}\SpecialCharTok{$}\NormalTok{title) }\SpecialCharTok{\%\textgreater{}\%}
  \FunctionTok{left\_join}\NormalTok{(d2f\_raw[, }\FunctionTok{c}\NormalTok{(}\StringTok{"title"}\NormalTok{, }\StringTok{"cluster\_km"}\NormalTok{)], }\AttributeTok{by =} \StringTok{"title"}\NormalTok{) }\SpecialCharTok{\%\textgreater{}\%}
\NormalTok{  tidyr}\SpecialCharTok{::}\FunctionTok{separate\_rows}\NormalTok{(genres, }\AttributeTok{sep =} \StringTok{"}\SpecialCharTok{\textbackslash{}\textbackslash{}}\StringTok{|"}\NormalTok{)}

\NormalTok{tabla\_generos }\OtherTok{\textless{}{-}}\NormalTok{ movies\_con\_cluster }\SpecialCharTok{\%\textgreater{}\%}
  \FunctionTok{group\_by}\NormalTok{(cluster\_km, genres) }\SpecialCharTok{\%\textgreater{}\%}
  \FunctionTok{summarise}\NormalTok{(}\AttributeTok{frecuencia =} \FunctionTok{n}\NormalTok{(), }\AttributeTok{.groups =} \StringTok{"drop"}\NormalTok{) }\SpecialCharTok{\%\textgreater{}\%}
  \FunctionTok{arrange}\NormalTok{(cluster\_km, }\FunctionTok{desc}\NormalTok{(frecuencia))}

\NormalTok{top\_generos }\OtherTok{\textless{}{-}}\NormalTok{ tabla\_generos }\SpecialCharTok{\%\textgreater{}\%}
  \FunctionTok{group\_by}\NormalTok{(cluster\_km) }\SpecialCharTok{\%\textgreater{}\%}
  \FunctionTok{slice\_max}\NormalTok{(frecuencia, }\AttributeTok{n =} \DecValTok{5}\NormalTok{)}

\FunctionTok{print}\NormalTok{(top\_generos)}
\end{Highlighting}
\end{Shaded}

\begin{verbatim}
## # A tibble: 15 x 3
## # Groups:   cluster_km [3]
##    cluster_km genres            frecuencia
##    <fct>      <chr>                  <int>
##  1 1          "Drama"                 1677
##  2 1          "Comedy"                1332
##  3 1          "Thriller"              1164
##  4 1          "Action"                1032
##  5 1          "Adventure"              688
##  6 2          "Adventure"              295
##  7 2          "Action"                 266
##  8 2          "Science Fiction"        151
##  9 2          "Fantasy"                139
## 10 2          "Comedy"                 117
## 11 3          "Drama"                   32
## 12 3          "Comedy"                  29
## 13 3          ""                        14
## 14 3          "Horror"                  14
## 15 3          "Thriller"                14
\end{verbatim}

\begin{Shaded}
\begin{Highlighting}[]
\FunctionTok{ggplot}\NormalTok{(top\_generos, }\FunctionTok{aes}\NormalTok{(}\AttributeTok{x =} \FunctionTok{reorder}\NormalTok{(genres, frecuencia), }\AttributeTok{y =}\NormalTok{ frecuencia, }\AttributeTok{fill =}\NormalTok{ cluster\_km)) }\SpecialCharTok{+}
  \FunctionTok{geom\_col}\NormalTok{(}\AttributeTok{show.legend =} \ConstantTok{FALSE}\NormalTok{) }\SpecialCharTok{+}
  \FunctionTok{facet\_wrap}\NormalTok{(}\SpecialCharTok{\textasciitilde{}}\NormalTok{cluster\_km, }\AttributeTok{scales =} \StringTok{"free\_y"}\NormalTok{) }\SpecialCharTok{+}
  \FunctionTok{coord\_flip}\NormalTok{() }\SpecialCharTok{+}
  \FunctionTok{scale\_fill\_brewer}\NormalTok{(}\AttributeTok{palette =} \StringTok{"Set2"}\NormalTok{) }\SpecialCharTok{+}
  \FunctionTok{labs}\NormalTok{(}\AttributeTok{title    =} \StringTok{"Top 5 Géneros por Cluster"}\NormalTok{,}
       \AttributeTok{subtitle =} \StringTok{"Distribución de géneros dentro de cada grupo de películas"}\NormalTok{,}
       \AttributeTok{x =} \StringTok{"Género"}\NormalTok{, }\AttributeTok{y =} \StringTok{"Frecuencia"}\NormalTok{) }\SpecialCharTok{+}
  \FunctionTok{theme\_bw}\NormalTok{()}
\end{Highlighting}
\end{Shaded}

\pandocbounded{\includegraphics[keepaspectratio]{Lab2_files/figure-latex/cluster_freq_genres-1.pdf}}

El Cluster 1 está dominado por Drama y Comedia, géneros de producción
accesible y presupuestos moderados. El Cluster 2 tiene Adventure y
Action como géneros principales, consistente con las grandes
producciones de Hollywood. El Cluster 4 comparte géneros con el Cluster
2 (Adventure, Action, Sci-Fi), confirmando que son películas de
franquicias masivas cuyo diferenciador es la popularidad viral.

\subsubsection{5.4 Perfil normalizado
comparativo}\label{perfil-normalizado-comparativo}

\begin{Shaded}
\begin{Highlighting}[]
\NormalTok{perfil\_norm }\OtherTok{\textless{}{-}}\NormalTok{ perfil }\SpecialCharTok{\%\textgreater{}\%}
  \FunctionTok{select}\NormalTok{(cluster\_km, budget\_media, revenue\_media, pop\_media, voteAvg\_media, runtime\_media) }\SpecialCharTok{\%\textgreater{}\%}
  \FunctionTok{mutate}\NormalTok{(}\FunctionTok{across}\NormalTok{(}\SpecialCharTok{{-}}\NormalTok{cluster\_km, }\SpecialCharTok{\textasciitilde{}} \FunctionTok{round}\NormalTok{((. }\SpecialCharTok{{-}} \FunctionTok{min}\NormalTok{(.)) }\SpecialCharTok{/}\NormalTok{ (}\FunctionTok{max}\NormalTok{(.) }\SpecialCharTok{{-}} \FunctionTok{min}\NormalTok{(.)), }\DecValTok{4}\NormalTok{)))}

\FunctionTok{cat}\NormalTok{(}\StringTok{"=== Perfil normalizado de clusters (0 = mínimo, 1 = máximo) ===}\SpecialCharTok{\textbackslash{}n}\StringTok{"}\NormalTok{)}
\end{Highlighting}
\end{Shaded}

\begin{verbatim}
## === Perfil normalizado de clusters (0 = mínimo, 1 = máximo) ===
\end{verbatim}

\begin{Shaded}
\begin{Highlighting}[]
\FunctionTok{print}\NormalTok{(perfil\_norm)}
\end{Highlighting}
\end{Shaded}

\begin{verbatim}
## # A tibble: 3 x 6
##   cluster_km budget_media revenue_media pop_media voteAvg_media runtime_media
##   <fct>             <dbl>         <dbl>     <dbl>         <dbl>         <dbl>
## 1 1                 0.199         0.126     0.139         0.929         0.840
## 2 2                 1             1         1             1             1    
## 3 3                 0             0         0             0             0
\end{verbatim}

\begin{Shaded}
\begin{Highlighting}[]
\NormalTok{perfil\_long }\OtherTok{\textless{}{-}}\NormalTok{ perfil\_norm }\SpecialCharTok{\%\textgreater{}\%}
\NormalTok{  tidyr}\SpecialCharTok{::}\FunctionTok{pivot\_longer}\NormalTok{(}\SpecialCharTok{{-}}\NormalTok{cluster\_km, }\AttributeTok{names\_to =} \StringTok{"variable"}\NormalTok{, }\AttributeTok{values\_to =} \StringTok{"valor"}\NormalTok{)}

\FunctionTok{ggplot}\NormalTok{(perfil\_long, }\FunctionTok{aes}\NormalTok{(}\AttributeTok{x =}\NormalTok{ variable, }\AttributeTok{y =}\NormalTok{ valor, }\AttributeTok{fill =}\NormalTok{ cluster\_km)) }\SpecialCharTok{+}
  \FunctionTok{geom\_col}\NormalTok{(}\AttributeTok{position =} \StringTok{"dodge"}\NormalTok{, }\AttributeTok{alpha =} \FloatTok{0.8}\NormalTok{) }\SpecialCharTok{+}
  \FunctionTok{scale\_fill\_brewer}\NormalTok{(}\AttributeTok{palette =} \StringTok{"Set2"}\NormalTok{) }\SpecialCharTok{+}
  \FunctionTok{labs}\NormalTok{(}\AttributeTok{title    =} \StringTok{"Perfil Comparativo de Clusters (valores normalizados)"}\NormalTok{,}
       \AttributeTok{subtitle =} \StringTok{"Identifica en qué dimensiones se diferencia cada grupo"}\NormalTok{,}
       \AttributeTok{x =} \StringTok{"Variable"}\NormalTok{, }\AttributeTok{y =} \StringTok{"Valor Normalizado (0–1)"}\NormalTok{, }\AttributeTok{fill =} \StringTok{"Cluster"}\NormalTok{) }\SpecialCharTok{+}
  \FunctionTok{theme\_bw}\NormalTok{() }\SpecialCharTok{+}
  \FunctionTok{theme}\NormalTok{(}\AttributeTok{axis.text.x =} \FunctionTok{element\_text}\NormalTok{(}\AttributeTok{angle =} \DecValTok{30}\NormalTok{, }\AttributeTok{hjust =} \DecValTok{1}\NormalTok{))}
\end{Highlighting}
\end{Shaded}

\pandocbounded{\includegraphics[keepaspectratio]{Lab2_files/figure-latex/cluster_perfil_comparativo-1.pdf}}

El Cluster 4 domina en popularidad (valor 1.0) con valores intermedios
en budget y revenue, confirmando que popularidad y éxito financiero no
son equivalentes. El Cluster 2 lidera en budget y revenue. El Cluster 3
tiene valores cercanos a cero en todas las variables. El Cluster 1
mantiene valores moderados en todas las dimensiones.

\subsubsection{5.5 Dispersión Budget vs Revenue y Popularidad vs
Calificación}\label{dispersiuxf3n-budget-vs-revenue-y-popularidad-vs-calificaciuxf3n}

\begin{Shaded}
\begin{Highlighting}[]
\FunctionTok{ggplot}\NormalTok{(d2f\_raw, }\FunctionTok{aes}\NormalTok{(}\AttributeTok{x =}\NormalTok{ budget, }\AttributeTok{y =}\NormalTok{ revenue, }\AttributeTok{color =}\NormalTok{ cluster\_km)) }\SpecialCharTok{+}
  \FunctionTok{geom\_point}\NormalTok{(}\AttributeTok{alpha =} \FloatTok{0.5}\NormalTok{, }\AttributeTok{size =} \FloatTok{1.5}\NormalTok{) }\SpecialCharTok{+}
  \FunctionTok{scale\_color\_brewer}\NormalTok{(}\AttributeTok{palette =} \StringTok{"Set2"}\NormalTok{) }\SpecialCharTok{+}
  \FunctionTok{scale\_x\_continuous}\NormalTok{(}\AttributeTok{labels =}\NormalTok{ scales}\SpecialCharTok{::}\NormalTok{comma) }\SpecialCharTok{+}
  \FunctionTok{scale\_y\_continuous}\NormalTok{(}\AttributeTok{labels =}\NormalTok{ scales}\SpecialCharTok{::}\NormalTok{comma) }\SpecialCharTok{+}
  \FunctionTok{labs}\NormalTok{(}\AttributeTok{title    =} \StringTok{"Budget vs Revenue por Cluster"}\NormalTok{,}
       \AttributeTok{subtitle =} \StringTok{"Cada color representa un grupo con características similares"}\NormalTok{,}
       \AttributeTok{x =} \StringTok{"Presupuesto (Budget)"}\NormalTok{, }\AttributeTok{y =} \StringTok{"Ingresos (Revenue)"}\NormalTok{, }\AttributeTok{color =} \StringTok{"Cluster"}\NormalTok{) }\SpecialCharTok{+}
  \FunctionTok{theme\_bw}\NormalTok{()}
\end{Highlighting}
\end{Shaded}

\pandocbounded{\includegraphics[keepaspectratio]{Lab2_files/figure-latex/cluster_scatter_revenue_budget-1.pdf}}

\begin{Shaded}
\begin{Highlighting}[]
\FunctionTok{ggplot}\NormalTok{(d2f\_raw, }\FunctionTok{aes}\NormalTok{(}\AttributeTok{x =}\NormalTok{ popularity, }\AttributeTok{y =}\NormalTok{ voteAvg, }\AttributeTok{color =}\NormalTok{ cluster\_km)) }\SpecialCharTok{+}
  \FunctionTok{geom\_point}\NormalTok{(}\AttributeTok{alpha =} \FloatTok{0.5}\NormalTok{, }\AttributeTok{size =} \FloatTok{1.5}\NormalTok{) }\SpecialCharTok{+}
  \FunctionTok{scale\_color\_brewer}\NormalTok{(}\AttributeTok{palette =} \StringTok{"Set2"}\NormalTok{) }\SpecialCharTok{+}
  \FunctionTok{labs}\NormalTok{(}\AttributeTok{title    =} \StringTok{"Popularidad vs Calificación Promedio por Cluster"}\NormalTok{,}
       \AttributeTok{subtitle =} \StringTok{"¿Los clusters más populares son también los mejor calificados?"}\NormalTok{,}
       \AttributeTok{x =} \StringTok{"Popularidad"}\NormalTok{, }\AttributeTok{y =} \StringTok{"Calificación Promedio (voteAvg)"}\NormalTok{, }\AttributeTok{color =} \StringTok{"Cluster"}\NormalTok{) }\SpecialCharTok{+}
  \FunctionTok{theme\_bw}\NormalTok{()}
\end{Highlighting}
\end{Shaded}

\pandocbounded{\includegraphics[keepaspectratio]{Lab2_files/figure-latex/cluster_scatter_popularity_vote-1.pdf}}

En Budget vs Revenue, el Cluster 2 ocupa la esquina superior derecha
combinando grandes presupuestos con grandes ingresos. El Cluster 1 forma
una nube densa en el rango bajo-medio. El Cluster 4 no alcanza los
ingresos más altos, coherente con su naturaleza viral más que
financiera. En el gráfico de popularidad vs calificación, el Cluster 4
se separa drásticamente en el eje de popularidad, pero sus
calificaciones (7.3--8.5) son similares a las de los otros clusters.
Esto indica que la popularidad en TMDB responde más a franquicias
reconocibles que a la calidad objetiva del film.

\begin{center}\rule{0.5\linewidth}{0.5pt}\end{center}

\section{PCA}\label{pca}

\subsection{Paso 0 - Análisis de
viabilidad}\label{paso-0---anuxe1lisis-de-viabilidad}

\subsubsection{KMO}\label{kmo}

\begin{Shaded}
\begin{Highlighting}[]
\NormalTok{datos\_cuant\_clean }\OtherTok{\textless{}{-}}\NormalTok{ datos\_cuant[, }\FunctionTok{sapply}\NormalTok{(datos\_cuant, is.numeric)]}
\NormalTok{datos\_cuant\_clean }\OtherTok{\textless{}{-}} \FunctionTok{na.omit}\NormalTok{(datos\_cuant\_clean)}
\FunctionTok{KMO}\NormalTok{(datos\_cuant\_clean)}
\end{Highlighting}
\end{Shaded}

\begin{verbatim}
## Kaiser-Meyer-Olkin factor adequacy
## Call: KMO(r = datos_cuant_clean)
## Overall MSA =  0.84
## MSA for each item = 
##                    budget                   revenue                   runtime 
##                      0.85                      0.75                      0.92 
##                   voteAvg                 voteCount              genresAmount 
##                      0.88                      0.85                      0.92 
##        productionCoAmount productionCountriesAmount              actorsAmount 
##                      0.89                      0.57                      0.82 
##           castWomenAmount             castMenAmount               releaseYear 
##                      0.81                      0.49                      0.87 
##                popularity 
##                      0.90
\end{verbatim}

\subsubsection{Esfericidad de Bartlett}\label{esfericidad-de-bartlett}

\begin{Shaded}
\begin{Highlighting}[]
\FunctionTok{cortest.bartlett}\NormalTok{(}\FunctionTok{cor}\NormalTok{(datos\_cuant\_clean), }\AttributeTok{n =} \FunctionTok{nrow}\NormalTok{(datos\_cuant\_clean))}
\end{Highlighting}
\end{Shaded}

\begin{verbatim}
## $chisq
## [1] 115646.6
## 
## $p.value
## [1] 0
## 
## $df
## [1] 78
\end{verbatim}

\subsection{Visualización de
correlaciones}\label{visualizaciuxf3n-de-correlaciones}

\begin{Shaded}
\begin{Highlighting}[]
\NormalTok{matriz }\OtherTok{\textless{}{-}} \FunctionTok{cor}\NormalTok{(datos\_cuant\_clean, }\AttributeTok{use =} \StringTok{"pairwise.complete.obs"}\NormalTok{)}
\FunctionTok{corrplot}\NormalTok{(matriz,}
         \AttributeTok{method =} \StringTok{"color"}\NormalTok{,}
         \AttributeTok{type =} \StringTok{"upper"}\NormalTok{,}
         \AttributeTok{addCoef.col =} \StringTok{"black"}\NormalTok{,}
         \AttributeTok{tl.col =} \StringTok{"black"}\NormalTok{,}
         \AttributeTok{tl.srt =} \DecValTok{45}\NormalTok{)}
\end{Highlighting}
\end{Shaded}

\pandocbounded{\includegraphics[keepaspectratio]{Lab2_files/figure-latex/unnamed-chunk-5-1.pdf}}

\subsection{Paso 1 - Estandarización}\label{paso-1---estandarizaciuxf3n}

\begin{Shaded}
\begin{Highlighting}[]
\NormalTok{datos\_std }\OtherTok{\textless{}{-}} \FunctionTok{scale}\NormalTok{(datos\_cuant\_clean)}
\FunctionTok{apply}\NormalTok{(datos\_std, }\DecValTok{2}\NormalTok{, mean)}
\end{Highlighting}
\end{Shaded}

\begin{verbatim}
##                    budget                   revenue                   runtime 
##              5.057293e-18             -3.322679e-17             -6.086636e-17 
##                   voteAvg                 voteCount              genresAmount 
##             -3.906434e-17              2.896058e-17              1.045531e-17 
##        productionCoAmount productionCountriesAmount              actorsAmount 
##              5.644135e-17             -7.570402e-17              5.605938e-18 
##           castWomenAmount             castMenAmount               releaseYear 
##              7.311016e-18             -5.968415e-18             -3.106698e-15 
##                popularity 
##              5.162070e-19
\end{verbatim}

\begin{Shaded}
\begin{Highlighting}[]
\FunctionTok{apply}\NormalTok{(datos\_std, }\DecValTok{2}\NormalTok{, sd)}
\end{Highlighting}
\end{Shaded}

\begin{verbatim}
##                    budget                   revenue                   runtime 
##                         1                         1                         1 
##                   voteAvg                 voteCount              genresAmount 
##                         1                         1                         1 
##        productionCoAmount productionCountriesAmount              actorsAmount 
##                         1                         1                         1 
##           castWomenAmount             castMenAmount               releaseYear 
##                         1                         1                         1 
##                popularity 
##                         1
\end{verbatim}

\subsection{Paso 2 - PCA}\label{paso-2---pca}

\begin{Shaded}
\begin{Highlighting}[]
\NormalTok{compPrinc }\OtherTok{\textless{}{-}} \FunctionTok{prcomp}\NormalTok{(datos\_std)}
\FunctionTok{summary}\NormalTok{(compPrinc)}
\end{Highlighting}
\end{Shaded}

\begin{verbatim}
## Importance of components:
##                           PC1    PC2    PC3     PC4     PC5     PC6     PC7
## Standard deviation     2.2118 1.3154 1.1649 0.97870 0.87276 0.85329 0.76082
## Proportion of Variance 0.3763 0.1331 0.1044 0.07368 0.05859 0.05601 0.04453
## Cumulative Proportion  0.3763 0.5094 0.6138 0.68749 0.74608 0.80209 0.84662
##                            PC8     PC9    PC10    PC11    PC12    PC13
## Standard deviation     0.72903 0.64735 0.62165 0.55580 0.43058 0.40333
## Proportion of Variance 0.04088 0.03224 0.02973 0.02376 0.01426 0.01251
## Cumulative Proportion  0.88750 0.91974 0.94946 0.97323 0.98749 1.00000
\end{verbatim}

\subsection{Scree Plot}\label{scree-plot}

\begin{Shaded}
\begin{Highlighting}[]
\FunctionTok{fviz\_eig}\NormalTok{(compPrinc)}
\end{Highlighting}
\end{Shaded}

\begin{verbatim}
## Warning in geom_bar(stat = "identity", fill = barfill, color = barcolor, :
## Ignoring empty aesthetic: `width`.
\end{verbatim}

\pandocbounded{\includegraphics[keepaspectratio]{Lab2_files/figure-latex/unnamed-chunk-8-1.pdf}}

\subsection{Visualización de
variables}\label{visualizaciuxf3n-de-variables}

\begin{Shaded}
\begin{Highlighting}[]
\FunctionTok{fviz\_pca\_var}\NormalTok{(compPrinc,}
             \AttributeTok{col.var =} \StringTok{"cos2"}\NormalTok{,}
             \AttributeTok{gradient.cols =} \FunctionTok{c}\NormalTok{(}\StringTok{"\#00AFBB"}\NormalTok{, }\StringTok{"\#E7B800"}\NormalTok{, }\StringTok{"\#FC4E07"}\NormalTok{),}
             \AttributeTok{repel =} \ConstantTok{TRUE}\NormalTok{)}
\end{Highlighting}
\end{Shaded}

\begin{verbatim}
## Warning: Using `size` aesthetic for lines was deprecated in ggplot2 3.4.0.
## i Please use `linewidth` instead.
## i The deprecated feature was likely used in the ggpubr package.
##   Please report the issue at <https://github.com/kassambara/ggpubr/issues>.
## This warning is displayed once per session.
## Call `lifecycle::last_lifecycle_warnings()` to see where this warning was
## generated.
\end{verbatim}

\pandocbounded{\includegraphics[keepaspectratio]{Lab2_files/figure-latex/unnamed-chunk-9-1.pdf}}

\section{Algoritmo A Priori}\label{algoritmo-a-priori}

El objetivo del algoritmo A Priori es descubrir patrones frecuentes de
co-ocurrencia entre características categóricas de las películas.

Se analizarán principalmente:

\begin{itemize}
\tightlist
\item
  Géneros
\item
  País de producción
\item
  Idioma original
\end{itemize}

Buscamos reglas del tipo:

Si una película pertenece al género X → entonces también suele
pertenecer al género Y.

\begin{center}\rule{0.5\linewidth}{0.5pt}\end{center}

\subsection{Preparación de datos}\label{preparaciuxf3n-de-datos}

\begin{Shaded}
\begin{Highlighting}[]
\FunctionTok{library}\NormalTok{(arules)}
\FunctionTok{library}\NormalTok{(arulesViz)}

\NormalTok{movies\_apriori }\OtherTok{\textless{}{-}}\NormalTok{ movies }\SpecialCharTok{\%\textgreater{}\%}
  \FunctionTok{select}\NormalTok{(title, genres, productionCountry, originalLanguage) }\SpecialCharTok{\%\textgreater{}\%}
  \FunctionTok{filter}\NormalTok{(}\SpecialCharTok{!}\FunctionTok{is.na}\NormalTok{(genres))}

\NormalTok{movies\_apriori }\OtherTok{\textless{}{-}}\NormalTok{ movies\_apriori }\SpecialCharTok{\%\textgreater{}\%}
\NormalTok{  tidyr}\SpecialCharTok{::}\FunctionTok{separate\_rows}\NormalTok{(genres, }\AttributeTok{sep =} \StringTok{"}\SpecialCharTok{\textbackslash{}\textbackslash{}}\StringTok{|"}\NormalTok{)}

\NormalTok{transactions\_list }\OtherTok{\textless{}{-}}\NormalTok{ movies\_apriori }\SpecialCharTok{\%\textgreater{}\%}
  \FunctionTok{group\_by}\NormalTok{(title) }\SpecialCharTok{\%\textgreater{}\%}
  \FunctionTok{summarise}\NormalTok{(}\AttributeTok{items =} \FunctionTok{list}\NormalTok{(}\FunctionTok{unique}\NormalTok{(}\FunctionTok{c}\NormalTok{(genres,}
                                   \FunctionTok{paste0}\NormalTok{(}\StringTok{"Country\_"}\NormalTok{, productionCountry),}
                                   \FunctionTok{paste0}\NormalTok{(}\StringTok{"Lang\_"}\NormalTok{, originalLanguage)))))}

\NormalTok{transacciones }\OtherTok{\textless{}{-}} \FunctionTok{as}\NormalTok{(transactions\_list}\SpecialCharTok{$}\NormalTok{items, }\StringTok{"transactions"}\NormalTok{)}

\FunctionTok{summary}\NormalTok{(transacciones)}
\end{Highlighting}
\end{Shaded}

\begin{verbatim}
## transactions as itemMatrix in sparse format with
##  19386 rows (elements/itemsets/transactions) and
##  1521 columns (items) and a density of 0.002693743 
## 
## most frequent items:
##                          Lang_en                            Drama 
##                            11664                             6431 
## Country_United States of America                           Comedy 
##                             4881                             4766 
##                         Country_                          (Other) 
##                             3871                            47815 
## 
## element (itemset/transaction) length distribution:
## sizes
##    3    4    5    6    7    8    9   10   11   12   13 
## 8061 5074 3844 1571  539  201   60   21   10    3    2 
## 
##    Min. 1st Qu.  Median    Mean 3rd Qu.    Max. 
##   3.000   3.000   4.000   4.097   5.000  13.000 
## 
## includes extended item information - examples:
##      labels
## 1          
## 2    Action
## 3 Adventure
\end{verbatim}

\begin{center}\rule{0.5\linewidth}{0.5pt}\end{center}

\subsection{Items más frecuentes}\label{items-muxe1s-frecuentes}

\begin{Shaded}
\begin{Highlighting}[]
\FunctionTok{itemFrequencyPlot}\NormalTok{(transacciones,}
                  \AttributeTok{topN =} \DecValTok{15}\NormalTok{,}
                  \AttributeTok{type =} \StringTok{"absolute"}\NormalTok{,}
                  \AttributeTok{col =} \StringTok{"steelblue"}\NormalTok{,}
                  \AttributeTok{main =} \StringTok{"Items más frecuentes"}\NormalTok{)}
\end{Highlighting}
\end{Shaded}

\pandocbounded{\includegraphics[keepaspectratio]{Lab2_files/figure-latex/items_frecuentes-1.pdf}}

\begin{center}\rule{0.5\linewidth}{0.5pt}\end{center}

\subsection{Generación de reglas (Primer
intento)}\label{generaciuxf3n-de-reglas-primer-intento}

\begin{Shaded}
\begin{Highlighting}[]
\NormalTok{reglas1 }\OtherTok{\textless{}{-}} \FunctionTok{apriori}\NormalTok{(transacciones,}
                   \AttributeTok{parameter =} \FunctionTok{list}\NormalTok{(}\AttributeTok{supp =} \FloatTok{0.05}\NormalTok{,}
                                    \AttributeTok{conf =} \FloatTok{0.7}\NormalTok{,}
                                    \AttributeTok{minlen =} \DecValTok{2}\NormalTok{))}
\end{Highlighting}
\end{Shaded}

\begin{verbatim}
## Apriori
## 
## Parameter specification:
##  confidence minval smax arem  aval originalSupport maxtime support minlen
##         0.7    0.1    1 none FALSE            TRUE       5    0.05      2
##  maxlen target  ext
##      10  rules TRUE
## 
## Algorithmic control:
##  filter tree heap memopt load sort verbose
##     0.1 TRUE TRUE  FALSE TRUE    2    TRUE
## 
## Absolute minimum support count: 969 
## 
## set item appearances ...[0 item(s)] done [0.00s].
## set transactions ...[1521 item(s), 19386 transaction(s)] done [0.01s].
## sorting and recoding items ... [20 item(s)] done [0.00s].
## creating transaction tree ... done [0.00s].
## checking subsets of size 1 2 3 done [0.00s].
## writing ... [12 rule(s)] done [0.00s].
## creating S4 object  ... done [0.00s].
\end{verbatim}

\begin{Shaded}
\begin{Highlighting}[]
\FunctionTok{summary}\NormalTok{(reglas1)}
\end{Highlighting}
\end{Shaded}

\begin{verbatim}
## set of 12 rules
## 
## rule length distribution (lhs + rhs):sizes
## 2 3 
## 8 4 
## 
##    Min. 1st Qu.  Median    Mean 3rd Qu.    Max. 
##   2.000   2.000   2.000   2.333   3.000   3.000 
## 
## summary of quality measures:
##     support          confidence        coverage            lift      
##  Min.   :0.05937   Min.   :0.7081   Min.   :0.06257   Min.   :1.177  
##  1st Qu.:0.06511   1st Qu.:0.7592   1st Qu.:0.07930   1st Qu.:1.262  
##  Median :0.08235   Median :0.7764   Median :0.09115   Median :1.290  
##  Mean   :0.09671   Mean   :0.8568   Mean   :0.11306   Mean   :1.424  
##  3rd Qu.:0.09727   3rd Qu.:0.9986   3rd Qu.:0.12942   3rd Qu.:1.660  
##  Max.   :0.25132   Max.   :1.0000   Max.   :0.25178   Max.   :1.662  
##      count     
##  Min.   :1151  
##  1st Qu.:1262  
##  Median :1596  
##  Mean   :1875  
##  3rd Qu.:1886  
##  Max.   :4872  
## 
## mining info:
##           data ntransactions support confidence
##  transacciones         19386    0.05        0.7
##                                                                                  call
##  apriori(data = transacciones, parameter = list(supp = 0.05, conf = 0.7, minlen = 2))
\end{verbatim}

\begin{Shaded}
\begin{Highlighting}[]
\FunctionTok{inspect}\NormalTok{(}\FunctionTok{head}\NormalTok{(}\FunctionTok{sort}\NormalTok{(reglas1, }\AttributeTok{by =} \StringTok{"lift"}\NormalTok{), }\DecValTok{10}\NormalTok{))}
\end{Highlighting}
\end{Shaded}

\begin{verbatim}
##      lhs                                    rhs          support confidence   coverage     lift count
## [1]  {Country_United States of America,                                                              
##       Drama}                             => {Lang_en} 0.08882699  1.0000000 0.08882699 1.662037  1722
## [2]  {Comedy,                                                                                        
##       Country_United States of America}  => {Lang_en} 0.09341793  0.9994481 0.09346951 1.661120  1811
## [3]  {Action,                                                                                        
##       Country_United States of America}  => {Lang_en} 0.06251934  0.9991756 0.06257093 1.660667  1212
## [4]  {Country_United States of America,                                                              
##       Thriller}                          => {Lang_en} 0.06597545  0.9984387 0.06607861 1.659442  1279
## [5]  {Country_United States of America}  => {Lang_en} 0.25131538  0.9981561 0.25177963 1.658972  4872
## [6]  {Thriller}                          => {Lang_en} 0.13282781  0.7786513 0.17058702 1.294147  2575
## [7]  {Family}                            => {Lang_en} 0.06839988  0.7740806 0.08836274 1.286551  1326
## [8]  {Science Fiction}                   => {Lang_en} 0.06061075  0.7684761 0.07887135 1.277236  1175
## [9]  {Horror}                            => {Lang_en} 0.09259259  0.7631803 0.12132467 1.268434  1795
## [10] {Crime}                             => {Lang_en} 0.05937274  0.7474026 0.07943877 1.242211  1151
\end{verbatim}

\begin{center}\rule{0.5\linewidth}{0.5pt}\end{center}

\subsection{Paso 5: Segundo intento (más
flexible)}\label{paso-5-segundo-intento-muxe1s-flexible}

\begin{Shaded}
\begin{Highlighting}[]
\NormalTok{reglas2 }\OtherTok{\textless{}{-}} \FunctionTok{apriori}\NormalTok{(transacciones,}
                   \AttributeTok{parameter =} \FunctionTok{list}\NormalTok{(}\AttributeTok{supp =} \FloatTok{0.02}\NormalTok{,}
                                    \AttributeTok{conf =} \FloatTok{0.6}\NormalTok{,}
                                    \AttributeTok{minlen =} \DecValTok{2}\NormalTok{))}
\end{Highlighting}
\end{Shaded}

\begin{verbatim}
## Apriori
## 
## Parameter specification:
##  confidence minval smax arem  aval originalSupport maxtime support minlen
##         0.6    0.1    1 none FALSE            TRUE       5    0.02      2
##  maxlen target  ext
##      10  rules TRUE
## 
## Algorithmic control:
##  filter tree heap memopt load sort verbose
##     0.1 TRUE TRUE  FALSE TRUE    2    TRUE
## 
## Absolute minimum support count: 387 
## 
## set item appearances ...[0 item(s)] done [0.00s].
## set transactions ...[1521 item(s), 19386 transaction(s)] done [0.01s].
## sorting and recoding items ... [29 item(s)] done [0.00s].
## creating transaction tree ... done [0.00s].
## checking subsets of size 1 2 3 4 done [0.00s].
## writing ... [82 rule(s)] done [0.00s].
## creating S4 object  ... done [0.00s].
\end{verbatim}

\begin{Shaded}
\begin{Highlighting}[]
\FunctionTok{summary}\NormalTok{(reglas2)}
\end{Highlighting}
\end{Shaded}

\begin{verbatim}
## set of 82 rules
## 
## rule length distribution (lhs + rhs):sizes
##  2  3  4 
## 19 46 17 
## 
##    Min. 1st Qu.  Median    Mean 3rd Qu.    Max. 
##   2.000   3.000   3.000   2.976   3.000   4.000 
## 
## summary of quality measures:
##     support          confidence        coverage            lift       
##  Min.   :0.02012   Min.   :0.6143   Min.   :0.02094   Min.   : 1.021  
##  1st Qu.:0.02294   1st Qu.:0.6995   1st Qu.:0.03174   1st Qu.: 1.269  
##  Median :0.02902   Median :0.7713   Median :0.03722   Median : 1.658  
##  Mean   :0.04169   Mean   :0.8158   Mean   :0.05199   Mean   : 3.516  
##  3rd Qu.:0.04343   3rd Qu.:0.9962   3rd Qu.:0.05674   3rd Qu.: 2.517  
##  Max.   :0.25132   Max.   :1.0000   Max.   :0.25178   Max.   :28.627  
##      count       
##  Min.   : 390.0  
##  1st Qu.: 444.8  
##  Median : 562.5  
##  Mean   : 808.3  
##  3rd Qu.: 842.0  
##  Max.   :4872.0  
## 
## mining info:
##           data ntransactions support confidence
##  transacciones         19386    0.02        0.6
##                                                                                  call
##  apriori(data = transacciones, parameter = list(supp = 0.02, conf = 0.6, minlen = 2))
\end{verbatim}

\begin{Shaded}
\begin{Highlighting}[]
\NormalTok{reglas\_ordenadas }\OtherTok{\textless{}{-}} \FunctionTok{sort}\NormalTok{(reglas2, }\AttributeTok{by =} \StringTok{"lift"}\NormalTok{, }\AttributeTok{decreasing =} \ConstantTok{TRUE}\NormalTok{)}

\FunctionTok{inspect}\NormalTok{(}\FunctionTok{head}\NormalTok{(reglas\_ordenadas, }\DecValTok{15}\NormalTok{))}
\end{Highlighting}
\end{Shaded}

\begin{verbatim}
##      lhs                                    rhs                support confidence   coverage      lift count
## [1]  {Animation,                                                                                            
##       Lang_ja}                           => {Country_Japan} 0.02285154  0.9022403 0.02532756 28.626565   443
## [2]  {Country_Japan}                     => {Lang_ja}       0.03125967  0.9918167 0.03151759 22.202492   606
## [3]  {Lang_ja}                           => {Country_Japan} 0.03125967  0.6997691 0.04467141 22.202492   606
## [4]  {Animation,                                                                                            
##       Country_Japan}                     => {Lang_ja}       0.02285154  0.9910515 0.02305788 22.185362   443
## [5]  {Country_FR}                        => {Lang_fr}       0.02011761  0.9112150 0.02207779 16.146996   390
## [6]  {Animation,                                                                                            
##       Comedy,                                                                                               
##       Lang_en}                           => {Family}        0.02068503  0.8371608 0.02470855  9.474138   401
## [7]  {Animation,                                                                                            
##       Country_United States of America,                                                                     
##       Lang_en}                           => {Family}        0.02537914  0.7639752 0.03321985  8.645897   492
## [8]  {Animation,                                                                                            
##       Country_United States of America}  => {Family}        0.02548231  0.7635240 0.03337460  8.640791   494
## [9]  {Animation,                                                                                            
##       Comedy}                            => {Family}        0.02316104  0.7093207 0.03265243  8.027374   449
## [10] {Adventure,                                                                                            
##       Animation}                         => {Family}        0.02218096  0.6574924 0.03373569  7.440833   430
## [11] {Animation,                                                                                            
##       Lang_en}                           => {Family}        0.03739812  0.6496416 0.05756732  7.351986   725
## [12] {Country_Japan}                     => {Animation}     0.02305788  0.7315876 0.03151759  7.181041   447
## [13] {Country_Japan,                                                                                        
##       Lang_ja}                           => {Animation}     0.02285154  0.7310231 0.03125967  7.175501   443
## [14] {Adventure,                                                                                            
##       Family}                            => {Animation}     0.02218096  0.6231884 0.03559270  6.117028   430
## [15] {,                                                                                                     
##       Lang_en}                           => {Country_}      0.03213659  0.6838639 0.04699268  3.424796   623
\end{verbatim}

\begin{center}\rule{0.5\linewidth}{0.5pt}\end{center}

\subsection{Filtrado de reglas
relevantes}\label{filtrado-de-reglas-relevantes}

\begin{Shaded}
\begin{Highlighting}[]
\NormalTok{reglas\_filtradas }\OtherTok{\textless{}{-}} \FunctionTok{sort}\NormalTok{(reglas1, }\AttributeTok{by =} \StringTok{"lift"}\NormalTok{, }\AttributeTok{decreasing =} \ConstantTok{TRUE}\NormalTok{)}

\FunctionTok{inspect}\NormalTok{(}\FunctionTok{head}\NormalTok{(reglas\_filtradas, }\DecValTok{10}\NormalTok{))}
\end{Highlighting}
\end{Shaded}

\begin{verbatim}
##      lhs                                    rhs          support confidence   coverage     lift count
## [1]  {Country_United States of America,                                                              
##       Drama}                             => {Lang_en} 0.08882699  1.0000000 0.08882699 1.662037  1722
## [2]  {Comedy,                                                                                        
##       Country_United States of America}  => {Lang_en} 0.09341793  0.9994481 0.09346951 1.661120  1811
## [3]  {Action,                                                                                        
##       Country_United States of America}  => {Lang_en} 0.06251934  0.9991756 0.06257093 1.660667  1212
## [4]  {Country_United States of America,                                                              
##       Thriller}                          => {Lang_en} 0.06597545  0.9984387 0.06607861 1.659442  1279
## [5]  {Country_United States of America}  => {Lang_en} 0.25131538  0.9981561 0.25177963 1.658972  4872
## [6]  {Thriller}                          => {Lang_en} 0.13282781  0.7786513 0.17058702 1.294147  2575
## [7]  {Family}                            => {Lang_en} 0.06839988  0.7740806 0.08836274 1.286551  1326
## [8]  {Science Fiction}                   => {Lang_en} 0.06061075  0.7684761 0.07887135 1.277236  1175
## [9]  {Horror}                            => {Lang_en} 0.09259259  0.7631803 0.12132467 1.268434  1795
## [10] {Crime}                             => {Lang_en} 0.05937274  0.7474026 0.07943877 1.242211  1151
\end{verbatim}

\begin{center}\rule{0.5\linewidth}{0.5pt}\end{center}

\subsection{Paso 7: Visualización de
reglas}\label{paso-7-visualizaciuxf3n-de-reglas}

\begin{Shaded}
\begin{Highlighting}[]
\FunctionTok{plot}\NormalTok{(reglas\_filtradas, }\AttributeTok{method =} \StringTok{"graph"}\NormalTok{)}
\end{Highlighting}
\end{Shaded}

\pandocbounded{\includegraphics[keepaspectratio]{Lab2_files/figure-latex/unnamed-chunk-14-1.pdf}}

\begin{center}\rule{0.5\linewidth}{0.5pt}\end{center}

\subsection{Interpretación de reglas}\label{interpretaciuxf3n-de-reglas}

Los resultados del algoritmo Apriori (soporte mínimo 0.02 y confianza
0.6) muestran que el país de producción es un fuerte predictor del
idioma. Por ejemplo, las reglas que combinan Estados Unidos con géneros
como Drama, Comedy, Action o Thriller implican casi siempre que la
película está en inglés (confianzas ≈ 1 y lift ≈ 1.66). Además, la regla
\{Country\_United States of America\} =\textgreater{} \{Lang\_en\} tiene
el mayor soporte (0.25), indicando que una cuarta parte del total son
películas estadounidenses en inglés.

También destacan asociaciones muy fuertes entre Japón y el idioma
japonés, con lifts superiores a 22, lo que refleja una relación casi
determinística entre país e idioma. Por otro lado, el género Animation
aparece frecuentemente vinculado con Family, con lifts entre 7 y 9,
mostrando que las películas animadas tienden a orientarse al público
familiar. En conjunto, las reglas revelan patrones claros entre país,
idioma y ciertos géneros.

\section{Seleccion de Algoritmo de Aprendizaje no
supervisado}\label{seleccion-de-algoritmo-de-aprendizaje-no-supervisado}

\section{Aprendizaje no supervisado:
SVD}\label{aprendizaje-no-supervisado-svd}

\begin{center}\rule{0.5\linewidth}{0.5pt}\end{center}

\section{Conclusiones y Hallazgos}\label{conclusiones-y-hallazgos}

\subsection{Hallazgos del Clustering}\label{hallazgos-del-clustering}

El análisis sobre 4,262 películas reveló 4 grupos naturales, validados
por Hopkins cercano a 1.0 y silueta promedio de 0.537 para K-Means.

Cluster 1 --- Producciones Convencionales (87\% del catálogo): La
columna vertebral de la industria. Presupuestos medianos de \$20M e
ingresos de \$44M (ratio \textasciitilde2.2x). Dominados por Drama,
Comedia y Thriller. Segmento altamente competitivo con márgenes
moderados.

Cluster 2 --- Blockbusters Comerciales (11\%): El grupo más rentable,
con ratio ingreso/presupuesto de \textasciitilde4:1 (\$540M vs \$130M).
Producciones de Action y Adventure que generan el mayor retorno
absoluto. La silueta baja (0.14) en algunas películas sugiere que la
frontera con el Cluster 1 es difusa.

Cluster 3 --- Producciones Marginales (2\%): Datos financieros casi
nulos, probablemente cortometrajes o registros incompletos en TMDB. No
aporta valor analítico directo pero es importante identificarlo para no
contaminar otros análisis.

Cluster 4 --- Fenómenos Virales (7 películas): El hallazgo más
sorprendente. Spider-Man: No Way Home (popularidad 8,444), Eternals
(11,475) y Sing 2 (6,056) forman un grupo único con popularidad viral
masiva desproporcionada respecto a sus cifras financieras, impulsada por
su pertenencia a franquicias del MCU y Sony.

Conclusión estratégica para CineVision Studios: La popularidad y el
éxito comercial son dimensiones independientes. El Cluster 4 demuestra
que pertenecer a franquicias con comunidades de fans activas genera un
engagement que supera incluso al de blockbusters de mayor presupuesto.
Invertir en propiedad intelectual reconocible y construcción de
universos narrativos a largo plazo puede ser tan rentable como competir
directamente en presupuesto con los grandes estudios.

\end{document}
